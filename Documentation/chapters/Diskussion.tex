\chapter{Diskussion}

\section{Interpretation der Resultate}
Warum funktionieren einige Versionen besser als andere

\subsection*{Ergebnisse des Trainings mit Schrifterkennung}
- Wurde KI besser? In welchen Kriterien?
- Warum nicht?
- läuft es besser auf Basisversion oder Physik

- Hilfreich für Speed? (Vergleich Speed Training, Basisversion Speed Test)

\subsection*{Ergebnisse der Physikalischen Umgebung}
- Vergleich zu Basisversion: Schlechter oder besser
- Vergleich durchschnittsspeed. Wie kann Verglichen werden (andere Skala)
- Probleme der Physikalischen Umgebung: Bump in Wand

\section{Vergleich zu menschlichem Zeichnen}
- Einziger Anhaltspunkt in Gifs, Videos

- Stiftbewegungen realistischer durch Physik?
Physikalische Entfernung von Realität? -> 2D Kein Druck auf Stift -> mögliche Erweiterung

- Geschwindigkeit der Bewegung
    (Möglich: Zwar schnell aber nicht realistisch, Springt hin und her, Physik prioritiert Geschwindigkeit in gerader Linie über Pixelgenauigkeit)

- Limitierungen
    Unfähig, andere Symbole zu zeichnen. Nur Zahlenwerte
    Sehr unterschiedliche Ergebnisse


\section{Selbstreflexion}
In der Selbstreflexion sollen verschiedene Aspekte in der Entwicklung dieses
Projektes betrachtet und reflektiert werden. Es wird ein besonderer Wert auf 3
Nachfolgende Teilgebeite dieser Arbeit gelegt, welche zu beginn des Projektes in
der Projektvereinbarung mit der Neuen Kantonsschule Aarau festgelegt wurden.

Allgemein betrachtet verlief das Projekt relativ reibungslos und ohne grosse
Schwierigkeiten. Teilweise gab es Verzögerungen, weil nicht so gute Ansätze oder
Technologien eingesetzt wurden. Dennoch wurden diese schnell entdeckt und
verbessert. Ausserdem bleiben diese Art von Problemen bei einer solchen Arbeit
nicht aus.

% BUG: Präteritum?

\subsection*{Verwendung von Git und GitHub}


\subsection*{Optimierung der KI}


\subsection*{Analyse der KI}
