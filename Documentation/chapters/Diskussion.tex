\chapter{Diskussion}
\label{chap:d}
Die Diskussion analyisiert die Resultate der Methode (siehe \nameref{chap:m}),
um daraus eine Antwort auf die Fragestellung zu bilden. Zu diesem Zweck werden
einige allgemeine Feststellungen getroffen  und die Unterfragen beantworten
(siehe \ref{chap:d_frage} \nameref{chap:d_frage}). Im zweiten Teil der Diskussion folgt ein Fazit,
ein Ausblick (siehe \nameref{chap:d_faz-aus}), und eine Selbstreflexion (siehe
\nameref{chap:d_reflex}). Dabei verschiebt sich der Fokus von der Fragestellung
weg und auf eine allgemeinere Betrachtung der Arbeit.


\section{Fragestellung und Unterfragen}
\label{chap:d_frage}
Die Fragestellung und die Unterfragen decken nicht alle Erkenntnisse aus den
Resultaten ab. Einige allgemeine Feststellungen geben Einblick, wie die
Resultate (siehe \nameref{chap:r}) zu verstehen sind. 

Die Grund-Basis Version und die Grund-Speed Version (siehe
\nameref{chap:m_auswert}) erreichen in allen Kriterien für alle Datensets die
beste Leistung. Die Resultate zwischen den Versionen sind dabei fast
ununterscheidbar. Vor allem im Kriterium der Geschwindigkeit (siehe
\nameref{sub:m_eval_speed}) zeigt die Speed Variation keine Verbesserung. Unter
den Versionen, die auf der physikalischen Umgebung basieren, erreichen Ebenfalls
die Physik-Basis und die Phyisk-Speed Versionen die beste Leistung. Die
Grund-MNIST Version und die Physik-MNIST Version sind in allen Kriterien
schlechter als die Basis und Speed Versionen. Auch im Kriterium der
Erkennbarkeit (siehe \nameref{sub:m_eval_rec}) bringt die Variation keinen
Vorteil. Die Physik-MNIST-Speed Version erbringt die schlechteste Leistung. Eine
Erklärung dafür ist, dass diese Version eine Kombination von allen Variationen
(siehe \nameref{chap:m_var}) ist und somit von der besten Version, der
Grund-Basis Version am stärksten abweicht.

Die Bildersammlung (siehe \nameref{chap:r_bild}) zeigt, dass die KI in vielen
Fällen präzise der Linie folgt und selten willkürliche Sprünge begeht. Das ist
interessant, weil der KI nie explizit mitgeteilt wird, wie genau sie zeichnen
sollte.


\subsection{Beantwortung der Unterfragen}
\label{sub:d_frage_unter}
Insgesamt sechs Unterfragen werden beantwortet (siehe \nameref{chap:einleit}).
Diese Unterfragen weiten die Fragestellung aus und tragen zu der
schlussendlichen Antwort auf die Fragestellung bei. Die Antworten beruhen auf
den Resultaten, aber auch auf Erkenntnissen aus der Methode (siehe
\nameref{chap:m}) selbst.

\subsubsection*{Wie kann die Architektur einer KI aussehen, die das Nachzeichnen erlernt?}
\label{subsub:d_frage_unter_1}
Unter der Annahme, dass die KI dieser Arbeit das Nachzeichnen erlernt, (siehe
\nameref{sub:d_frage_frag}), kann die Architektur genau so aussehen, wie sie in
dieser Arbeit beschrieben ist (siehe \nameref{sub:m_grund_dood}).

\subsubsection*{Wie lässt sich die Leistung der KI in dieser Aufgabe beurteilen?}
\label{subsub:d_frage_unter_2}
Die Leistung der KI lässt sich durch die definierten Kriterien (siehe
\nameref{chap:m_eval}) beurteilen. Das Kriterium Die Übereinstimmung ist ein
objektiver und Absoluter Wert, und somit das Aussagekräftigste Kriterium.
Ausserdem ist der maximale Wert des Kriteriums, unabhängig vom gezeichneten
Bild, gleich $1$. Dadurch ist das Kriterium geeignet für Vergleiche zwischen
Versionen der KI.

die Kriterien der Erkennbarkeit und der Geschwindigkeit sind an subjektive
Annahmen gebunden. Zum Beispiel wird für das Kriterium der Geschwindigkeit ein
subjektiver Punkt der Fertigstellung definiert (siehe
\nameref{sub:m_eval_speed}). Dadurch sinkt ihre Aussagekraft. Allerdings
verändern sich die Annahmen nicht und die Kriterien sind in jedem Fall durch
einen Zahlenwert repräsentiert. Somit eignen sich auch diese Kriterien für
Vergleiche zwischen Versionen der KI. Aus der Annahme heraus, dass für Menschen
beim Nachzeichnen Erkennbarkeit wichtiger als absolute Genauigkeit ist, ergibt
sich das Kriterium der Erkennbarkeit als besonders wichtiges. Aus diesem Grund
ist das Kriterium in der Fragestellung (siehe \nameref{sub:d_frage_frag})
vermerkt.


\subsubsection*{Wie lässt sich die Leistung der KI verbessern?}
\label{subsub:d_frage_unter_3}
Bezogen auf die definierten Kriterien erreicht die Grundversion Werte, die durch
die implementierten Variationen nicht oder nur marginal verbessert werden. Die
Variationen der KI sind somit insgesamt ein gescheiterter Versuch der
Verbesserung der Leistung. Die Grundversion erfuhr allerdings in dessen
Entwicklung signifikante Verbesserungen. Die grössten Verbesserungen stammen aus
der Optimierung der Hyperparamter durch den Baysian Optimization Algorithmus
(siehe \nameref{sub:m_grund_data}). Zum Beispiel hat die Grösse des Replay
Buffers einen erheblichen Effekt auf die Leistung.

\subsubsection*{Welche Einflüsse haben Physiksimulationen auf die Leistung der KI?}
\label{subsub:d_frage_unter_4}
Bezogen auf die definierten Kriterien verschlechtert sich die Leistung der KI.
Alle Versionen, die auf der Grundumgebung basieren, erzielen höhere Werte als
die gleichen Versionen basierend auf der phyiskalischen Umgebung (siehe
\nameref{sub:m_var_phy}). Die physikalische Umgebung hat zum Ziel, die
Bewegungen der KI realistischer zu gestalten. In diesem Bereich kann der
Einfluss nicht objektiv bestimmt werden. Aus Beobachtungen der Bilder, welche in
der physikalischen Umgebung gezeichnet sind (siehe \nameref{chap:r_bild}), gehen
ebenfalls keine Erkenntnisse in diesem Bereich hervor. Die Bilder unterscheiden
sich kaum von denjenigen aus der Grundumgebung.

\subsubsection*{Wie ändert sich die Leistung der KI bei Strichbildern, die sich von den Trainingsdaten unterscheidenn}
\label{subsub:d_frage_unter_5}
In allen acht Versionen bleibt die Leistung der KI zwischen den drei Datensets
(siehe \autoref{tab:datasets}) vergleichbar. Die Tabellen {...} und {...} Zeigen
die Leistung der Grund-Basis Version und der Physik-Basis Version in den drei
definierten Kriterien, getestet auf die drei Datensets. Der Wert der
Übereinstimmung zwischen dem MNIST Datenset und dem EMNIST Datenset ist beinahe
identisch. Für beide Versionen ist der Wert der Übereinstimmung für das
QuickDraw Datenset niedriger. Insgesamt ist die KI in diesem Kriterium jedoch
kaum Beeinflusst durch die Wahl des Datensets. Die Analyse der anderen zwei
Kriterien führt zu einer ähnlichen Schlussfolgerung. Interessant ist, dass vor
allem die Grund-Basis Version eine viel höhere Geschwindigkeit im Zeichnen von
MNIST Zahlen hat, als im Zeichnen von EMNIST Buchstaben. Obwohl die Formen zu
grossem Teil ähnlich sind, scheint die KI durch das spezifische Training auf
MNIST Ziffern eine höhere Geschwindigkeit zu entwickeln. 

%todo neue Tabellen

\subsubsection*{Inwiefern lässt sich das Zeichnen der KI mit menschlichem Zeichnen vergleichen?}
\label{subsub:d_frage_unter_6}
Die Antwort auf diese Frage leitet sich nicht aus den objektiven Resultaten ab,
sondern basiert auf subjektiven Beobachtungen. Die Bewegungen in der
Physik-Version der künstlichen Intelligenz basieren grundsätzlich auf den selben
Gesetzen wie die Bewegungen in der echten Welt. Allerdings sind die Bewegungen
stark vereinfacht im Vergleich zu menschlichen Bewegungen Ausserdem ist für die
künstliche Intelligenz der Druck des Stiftes nicht veränderbar. Zumindest
Konzeptuell aber nähert die künstliche Intelligenz menschliches Zeichnen,
bezogen auf die physischen Einschränkungen, an. Einige menschliche Gewohnheiten
sind bei der künstlichen Intelligenz allerdings nicht beobachtbar. Zum Beispiel
beginnt die künstliche Intelligenz beim Zeichnen von Ziffern an zufälligen
Orten, während Menschen in der Regel für jede Ziffer an der selben Stelle
ansetzen einer Ziffer immer an der selben Stelle


\subsection{Beantwortung der Fragestellung}
\label{sub:d_frage_frag}
Die Fragestellung lautet: in wiefern kann eine künstliche Intelligenz lernen,
Strichbilder auf eine physische Weise nachzuzeichnen, sodass diese durch ein
automatisches System erkannt werden? (siehe \nameref{chap:einleit}) Diese Frage
hat mehrere Aspekte, die teilweise bereits durch die Unterfragen (siehe
\nameref{sub:d_frage_unter}) erfasst werden. Für die schlussendliche Antwort
folgt eine genauere Ausführung der Aspekte.

Die KI zeichnet durch Physiksimulationen und durch allgemeine Einschränkungen
der Bewegungsfreiheit auf eine annähernd physische Weise. Das Zeichnen ist nur
annähernd physisch, da alle Bewegungen simuliert und in keiner phyischen
Umgebung umgesetzt sind. Ausserdem sind die Simulationen nicht vollkommen
realitätsgetreu (siehe \ref{sub:d_frage_unter} \nameref{subsub:d_frage_unter_4})

Die künstliche intelligenz erlernt das Nachzeichnen bezogen auf die Kriterien,
nach denen es definiert ist, erfolgreich. Dafür sprechen die Werte der besten
Versionen für das Nachzeichnen von Ziffern, die teilweise an den Höchstwert
grenzen (siehe \nameref{chap:d_frage}). Die hohen Werte im Kriterium der
Erkennbarkeit bestätigen ausserdem, dass die Zeichnungen der KI in den meisten
Fällen von einem automatischen System erkannt werden.

Laut der Fragestellung soll die KI das Nachzeichnen von Strichbildern erlernen.
Damit ist implizit das Nachzeichnen von allen möglichen Arten von Strichbildern
gemeint. Die Leistung der KI kann nicht auf alle möglichen Strichbilder
überprüft werden, aber der Test mit drei verschiedenen Datensets ergibt
vielversprechende Resultate (siehe \nameref{chap:r_tab}). Die KI erlernt
erfolgreich das Nachzeichnen von Ziffern, Kleinbuchstaben und zehn zufälligen
Motiven aus dem QuickDraw Datenset. Durch die Vielfalt im QuickDraw Datenset
kann die Annahme getroffen werden, dass die KI zumindest einen grossen Teil an
Strichbildern nachzeichnen kann. 

Die zusammenfassende Antwort auf die Frage lautet somit: Eine künstliche
Intelligenz kann das Nachzeichnen von Strichbildern auf annähernd physiche Weise
in dem Sinne lernen, dass die fertige Zeichnung von einem automatischen System
grösstenteils erkannt wird, Die Übereinstimmung zwischen der Vorlage und der
Zeichnung gross ist und die Zeichnung nicht viel Zeit in Anspruch nimmt.

Diese Antwort bezieht sich auf die genau Frage, wie sie in der Einleitung steht.
Der nächste Abschnitt beurteilt die Frage durch die Erkenntnisse aus dieser
Arbeit und geht auf mögliche Erweiterungen ein.


\section{Fazit und Ausblick}
\label{chap:d_faz-aus}
Die Resultate erlauben eine positive Antwort auf die Fragestellung. Diese
Antwort setzt allerdings einige Annahmen vorraus, die weiter diskutiert werden
können. Die grösste Annahme bezieht sich auf die Definition des Nachzeichnens.
Diese Arbeit definiert Nachzeichnen durch drei Kriterien und durch physische
Rahmenbedingungen. Die Kriterien sind für eine künstliche Intelligenz sinnvoll
gewählt (siehe \nameref{subsub:d_frage_unter_2}), Allerdings wären auch ander
Kriterien möglich. Nachzeichnen ist eine menschliche Tätigkeit. Dieser
Menschliche Aspekt ist in den definierten Kriterien nicht enthalten.

Die physichen Rahmenbedingungen unterscheiden sich von denjenigen, die ein
Mensch erfährt. Das kommt daher, dass die phyischen Rahmenbedingungen für die KI
lediglich simuliert sind. Das verunmöglicht eine umfassende Antwort auf die
Frage, ob die künstliche Intelligenz auf eine physische Weise zeichnet. Dieses
Problem könnte mit einem Roboter gelöst werden, der die künstliche Intelligenz
in eine reale, phyische Umgebung überführt. Der Roboter könnte somit
verschiedenste Strichbilder auf einem echten Stück Papier, und somit
zwangsläufig auf physische Weise nachzeichnen. Aktuell sind die Bewegungen der
künstlichen Intelligenz in gewissen Belangen eingeschränkt. So ist
beispielsweise die Druckstärke nicht variierbar. Ausserdem zeichnet die
künstliche Intelligenz vorwiegend kleine Strichbilder. Experimente mit grösseren
Konstrukten, wie ganze Wörter, wären eine mögliche Erweiterung. 

Alles in allem sind eine Vielzahl an denkbaren Fragen und Ideen möglich, die auf
ReSketch, der künstlichen Intelligenz hinter dieser Arbeit, basieren.


\section{Selbstreflexion}
\label{chap:d_reflex}
Die Selbstreflexion gibt genauere Einblicke in die Vorangehensweise hinter
dieser Arbeit. Diese Dokumentation ist grundsätzlich eine Zusammenfassung der wichtigsten
Ereignisse. Viele Aspekte, wie auch die Arbeitsweise bleiben verschwiegen. Die
selbstreflexion geht näher auf drei wichitige Aspekte ein, die in der
zusammengefassten Dokumentation nicht genug betont sind.

\subsection{Optimierung der KI}
\label{sub:d_reflex_opti}
Insgesamt sind acht Versionen der KI präsentiert. Im Verlaufe des Projektes gab
es viele weitere Versuche, die Leistung der künstlichen Intelligenz zu
verbessern. Diese Versuche führten allerdings häufig dazu, dass die KI den
akkumulierten Reward (siehe \nameref{sub:t_rl_func}) nicht mehr maximieren
konnte. In der Dokumentation sind deswegen nur diejenigen Versuche vermerkt, die
tatsächlich funktionieren. Das Problem hinter den versuchten Variation liegt
darin, dass die Ursache hinter ihrem Scheitern oder ihrem Erfolg häufig nicht
erkennbar ist. Das macht die Optimierung der künstlichen Intelligenz allgemein
schwierig. 

Die Strategie hinter der Optimierung besteht in den meisten Fällen aus
wiederholtem Ausprobieren mit Anpassungen zwischen jedem Versuch. Hilfsmittel,
wie der Baysian Optimization Algorithmus (siehe \nameref{sub:t_ml_hyper}),
vereinfachen diese Aufgabe massgeblich. Tatsächlich ermöglichte der Baysian
Optimization Algorithmus eine Verbesserung der KI von ungefähr $20-30\%$
Übereinstimmung zu den Werten, die sie schlussendlich erreicht (siehe
\nameref{chap:r_tab}). Diese Strategie der Optimierung ist für einen Computer
sehr ressourcenintensiv. In den längsten Optimierungsarbeiten liefen die beiden
Computer, auf denen die Arbeit verrichtet wurde, zusammen länger als $48$
Stunden.

\subsection{Analyse der künstlichen Intelligenz}
\label{sub:d_reflex_analys}
Eine Analyse der künstlichen Intelligenz ist notwendig, um die Fragestellung und
die Unterfragen zu beantworten. Aber auch während der Entwicklung der
KI ist eine stetige Analyse nötig, um dise zu verstehen und
zu verbessern.

Die Analyse besteht hauptsächlich darin, die Leistung der künstlichen
Intelligenz zu beurteilen. Das geschieht mittels den Kriterien, die für diesen
Zweck definiert sind (siehe \nameref{chap:m_eval}). Die Kriterien sind dabei so
definiert, dass sie für jede mögliche Variation identisch bleiben. Der
durchnittliche akkumulierte Reward ist beispielsweise absichtlich kein
Kriterium. Der akkumulierte Reward  ist abhängig von der Reward-Function (siehe
\nameref{sub:t_rl_func}) und unterscheidet sich somit zwischen Variationen.
Dieser ist somit nicht für einen Vergleich zwischen Variationen geeignet.

% Abschnitt wäre Streichbar
Einzelne Variationen vergleichbar zu halten, ist ein allgemeines Ziel der 
Analyse. Deswegen basieren alle Variationen auf der gleichen Architektur der KI.
Die Funktionalität des Grundprogrammes (siehe \nameref{chap:m_grund} ist
gründlich getestet. Zum Beispiel sind die Testdaten darauf überprüft, dass sie
keine Datenpunkte aus den Trainingsdaten beinhalten. Diese Tests eliminieren
mögliche Fehlerquellen falls die künstliche Intelligenz unerwartetes Verhalten
aufzeigt.

Eine weitere Form von Analyse stammt aus der Sammlung von Daten über das
Lernverhalten der KI. So wird aus jedem Training ein Graph erstellt, der die
durchschnittliche Leistung der KI in jeder Episode erfasst (siehe
autoref{lernkurve}). Die Leistung ist dabei durch den akkumulierten Reward in
jeder Episode repräsentiert. Wie erwähnt können Versionen der KI nicht anhand
ihres akkumulierten Rewards verglichen werden. Der akkumulierte Reward zeigt
allerdings für einzelne Versionen am präzisesten, in wiefern diese ihren Reward
maximieren können.

%todo bild Lernkurve

\subsection{Verwendung von Git und GitHub}
\label{sub:d_reflex_git}
Die Verwendung von Git und Github (siehe \nameref{chap:t_git}) erleichtert die
Arbeit an einem Projekt von diesem Ausmass sehr. Die Programme ermöglichen
einfache Zusammenarbeit am Programmcode und an der Dokumentation. GitHub dient
dabei zusätzlich als Hilfsmiitel zur Organisation durch die integrierte Funktion
der Project Boards. Diese Funktion hätte allerdings zu grösserem Ausmass
Verwendung finden können.

Die Funktion der Branches und Commits von Git (siehe \nameref{sub:t_git_git})
werden durch die Arbeit hindurch konsequent verwendet. So liegt der Ursprung
jeder Variation der künstlichen Intelligenz auf einem eigenen Branch. Auch die
verschiedenen Kapitel der Dokumentation sind jeweils in einem eigenen Branch
verfasst. Die Verzweigung des Projektes in Branches ermöglicht eine bessere
Ordnung und Strukturierung des Projektes. Wenn die Arbeit an einem Branch
erledigt ist, wird dieser wieder mit dem Main Branch zusammengeführt. Für die
wichtigsten Branches wird dabei das Prinzip der Pull Request (siehe
\nameref{sub:t_git_gh}) angewendet. Die Pull Request muss für jeden Branch von
beiden Autoren akzeptiert werden.

Ein weiterer Vorteil von Git und Github ist die Zugänglichkeit des Projektes.
Das gesamte Projekt ist unter folgendem Link einsehbar:
\url{https://github.com/LarsZauberer/Nachzeichner-KI}. Im Projektordner sind
vortrainierte Variationen der künstlichen Intelligenz enthalten. Das Projekt auf
GitHub erfährt möglicherweise Erweiterungen, die in dieser Arbeit nicht mehr
erfasst sind.






