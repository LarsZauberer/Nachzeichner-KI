\chapter{Einleitung}\label{chap:einleit}
Der Computer ist ein Werkzeug, das dem Menschen Arbeit abnehmen kann. Um
komplizierte Aufgaben zu übernehmen, muss sich der Computer jedoch an
menschliches Verhalten, menschliches Urteilsvermögen und menschliche Intelligenz
annähern. Mit anderen Worten braucht der Computer, oder das steuernde
Computerprogramm, eine künstliche Intelligenz. Ein Intelligentes
Computerprogramm zu entwickeln ist komplex. Der fähigste und am weitesten
verbreitete Ansatz liefert Machine Learning. Diese Arbeit selbst ist eine
Untersuchung Im Bereich Machine Learning. Spezifischer ist die Arbeit im Bereich
Reinforcement Learning, einem Teilgebiet von Machine Learning.

Die Fragestellung der Untersuchung lautet: Kann eine künstliche Intelligenz
lernen, Strichbilder auf eine physische Weise nachzuzeichnen, sodass diese durch
ein automatisches System richtig erkannt werden können?

Für ein gegebenes Strichbild soll die künstliche Intelligenz (KI) erlernen, ein
möglichst gleiches Bild daneben zeichnen können. Die Frage ist, ob die KI das
Nachzeichnen genug gut lernen kann, damit die Zeichnung von einem automatischen
System richtig erkannt wird. Richtig erkannt heisst in diesem Fall vereinfacht,
dass eine zweite KI in der Zeichnung das selbe Motiv wie in der Vorlage erkennt.
Wenn das zutrifft, kann die künstliche Intelligenz erfolgreich nachzeichnen. Es
existieren allerdings weitere Kriterien, die die Leistung der KI bei der
Tätigkeit des Nachzeichnens beurteilen.

Nachzeichnen ist eine menschliche Tätigkeit. Menschen führen beim Zeichnen durch
gewisse Handbewegungen einen Stift, wodurch das Nachzeichnen mit physischen
Einschränkungen verbunden ist. Der Stift kann sich nicht teleportieren, sondern
sich nur mit einer bestimmten Geschwindigkeit fortbewegen. Die KI soll das
Nachzeichnen mit ähnlichen physischen Einschränkungen erlernen. Mit anderen
Worten soll die KI lernen, einen Stift zu führen.  Die
physischen Einschränkungen sind dabei jedoch simuliert und im Vergleich zu der
echten Welt vereinfacht. 

Die KI soll das Nachzeichnen von Strichbildern allgemein erlernen. Strichbilder
können Zahlen, Buchstaben, Formen, Symbole und allgemeine Kritzeleien sein.
Natürlich kann die KI nicht mit allen Arten von Strichbildern trainiert werden,
weil die Vielfalt zu gross ist. Daraus ergibt sich die Frage, wie gut die
künstliche Intelligenz Arten von Strichbildern nachzeichnet, die nicht im
Training enthalten waren.

Die vorangehenden Überlegungen sind in einer Sammlung an Unterfragen, die in
dieser Arbeit beantwortet werden, vertreten. Die Unterfragen lauten:
\begin{itemize}
    \item Wie kann die Architektur einer KI aussehen, die das Nachzeichnen erlernt?
    \item Nach welchen Kriterien lässt sich die Leistung der KI in dieser Aufgabe beurteilen?
    \item Wie lässt sich die Leistung der KI in dieser Aufgabe verbessern?
    \item Wie ändert sich die Leistung der KI für Strichbilder, die im Training nicht enthalten sind?
    \item Welche Einflüsse haben physische Einschränkungen auf die Leistungs der KI?
    \item Wie und in wiefern lässt sich die KI mit menschlichem Zeichnen vergleichen?
\end{itemize}


