\chapter{Einleitung}
Computerprogramme sollen den Menschen Arbeit abnehmen. Dafür müssen sie
allerdings erstmals diese Tätigkeiten nachahmen können. Viele mensch\hyp{}liche
Tätigkeiten sind aber oft zu komplex und besitzen keine klar definierten Regeln,
sodass sie durch einen simplen Algorithmus beschrieben werden können. Um dieses
Problem kümmert sich das Teilgebiet der `Künstlichen Intelligenz' (kurz: KI). KI
Technologien haben in den letzten Jahren einen grossen Fortschritt gemacht. Sie
werden schon in vielerlei Anwendungen eingesetzt.
%  nicht zu letzt wegen der immer billiger und besser werdenden Hardware

Diese Arbeit beschäftigt sich mit der Tätigkeit des menschlichen Nach\hyp{}zeichnens
und mit der Zielfrage: `Kann eine künstliche Intelligenz lernen, Strichbilder
(Bspw. Zahlen, Buchstaben) nachzuzeichnen, sodass sie durch ein automatisches
System erkannt werden können?'

Die Problematik bei dieser Aufgabe ist, dass es viele verschiedene Arten gibt
gewisse Symbole zu zeichnen und es keine genau definierten Regeln gibt, dass zum
Beispiel eine Zahl 5 als eine 5 erkannt wird.
% TODO: Später nochmals reviewen

Da der Begriff Strichbilder ein sehr grober Begriff ist, werden in dieser Arbeit
Zahlen aus dem Mnist Datenset verwendet. % TODO: Ref in Theorieteil
Trotzdem sind die verwendeten Methoden auch auf andere Strichbilder, wie
Buchstaben oder Piktogramme anwendbar. Allerdings können die Resultate
variieren.

Diese Arbeit fällt nicht nur in den Bereich der künstlichen Intelligenz, sondern
auch in das Themengebiet der Robotik. Ein Roboter ist laut der Definition
`Apparatur, die bestimmte Funktionen eines Menschen ausführen kann'
\cite{noauthor_duden_nodate-1} Das Nachzeichnen von Strichbildern ist ebenfalls
eine sehr menschliche Tätigkeit. Diese Arbeit beschränkt sich allerdings nur auf
das Computer\hyp{}programm, also die Anwendung der künstlichen Intelligenz. Der Bau
einer Apparatur, die dann selber Zeichnen könnte ist hier nicht das Ziel.

Folgend sind die Ziele/Unterfragen dieser Arbeit weiter präzisiert:
\begin{itemize}
    \item Wie kann die Architektur einer KI aussehen, die das Nachzeichnen
    erlernt?
    \item Wie lässt sich die Leistung der KI in dieser Aufgabe beurteilen?
    \item Wie lässt sich die Leistung von dem Ergebnis verbessern?
    \item Welche Einflüsse haben die Integration von einfachen physischen
    Rah\hyp{}menbedingungen auf die Leistungs der KI?
    \item Wie und in wiefern lässt sich die Leistung der KI mit menschlichem
    Zeichnen vergleichen?
\end{itemize}
% TODO: Prosa Text
