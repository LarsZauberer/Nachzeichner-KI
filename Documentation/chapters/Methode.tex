\chapter{Methode}
\label{chap:m}
Die Methode dieser Untersuchung besteht darin, die in der Fragestellung
beschriebene künstliche Intelligenz (KI) zu entwickeln und dessen Leistung
auszuwerten. Die Diskussion dieser Resultate führt schlussendlich zu einer
Antwort auf die Fragestellung. Die Entwicklung der KI besteht aus zwei Teilen.
Der eine Teil umfasst die Definition der Kriterien, nach denen die Leistung der
KI evaluiert wird (siehe \nameref{chap:m_eval}). Der andere Teil umfasst die
Entwicklung der KI (siehe \nameref{chap:m_grund}), zusammen mit verschiedenen
Variationen (siehe \nameref{chap:m_var}). Die Variationen haben jeweils einen
unterschiedlichen Fokus auf die definierten Kriterien. Die Auswertung (siehe
\nameref{chap:m_auswert}) bezieht sich ebenfalls auf die definierten Kriterien.
Die Leistung der KI wird dabei in einer Testumgebung für das Zeichnen von
verschiedenen Arten von Strichbildern erfasst.

\section{Grundprogramm}
\label{chap:m_grund}
Die KI ist abhängig von den Kriterien, die dessen Leistung definieren (siehe
\nameref{chap:m_eval}). Mit anderen Worten trainiert die KI auf diese Kriterien.
Das Ziel des Grundprogrammes ist, die allgemeine Trainingsumgebung für die KI
bereitzustellen. Dieses Grundprogramm ist unabhängig von einem spezifischen
Kriterium und kann stattdessen auf ein ausgewähltes Kriterium trainiert werden.
Reinforcement Learning Modelle mit einer undefinierten Reward Function (siehe
\nameref{\label{sub:t_rl_func}}) stellen diese Eigenschaften bereit. Eine Reward
function, basierend auf einem spezifischen Kriterium, ermöglicht das Training
auf dieses Kriterium. Das Grundprogramm ist in Python unter der Verwendung des
Keras Frameworks implementiert (siehe \nameref{chap:t_ml}). 

\subsection{Doodle-SDQ als Basis}
\label{sub:m_grund_dood}
Das Reinforcement Learning Modell des Grundprogrammes basiert auf Doodle-SDQ.
(siehe \nameref{sub:t_ver_dood}) Von Doodle-SDQ ist das neuronale Netz, bezogen
auf die Form des Inputs, des Outputs und den Hidden Layers, zu grossem Teil
übernommen. Die relevanten Anpassungen zwischen Doodle-SDQ und dem Grundprogramm
dieser Arbeit sind nachfolgend erläutert.

Bei der Umgebung handelt es sich, wie bei Doodle-SDQ, um eine Zeichenfläche,
worauf sich der Agent frei bewegen kann. Die Ziffern, die während dem Training
nachgezeichnet werden sollen, stammen aus dem MNIST Datenset (siehe
\nameref{chap:t_ml}) und haben somit eine Grösse von $28\times28$ Pixeln. Die
Fläche, worauf sich der Agent bewegen   
kann, hat somit auch eine Grösse von $28\times28$ Pixeln. Der global Stream
(siehe \nameref{sub:t_ver_dood}) des Inputs in das Neuronale Netz ändert sich
bis auf die neue Grösse der Bilder nicht. Die Pixel der Bilder, wie auch die
Zeichenfläche, nehmen den Wert von einem Bit an. Eine Null repräsentiert einen
schwarzen (nicht gezeichneten) Pixel an dieser Stelle im Bild und eine Eins
einen weissen (gezeichneten) Pixel. Die genaue Architektur des neuronalen Netzes
ist im Schema autoref{architecture} angegeben. Jeder Block in der Abbildung
repräsentiert Eine Layer des neuronalen Netzes, wobei die Form des Inputs und
die Form des Outputs von jeder Layer angegeben ist. 

%TODO Bild architecture

Der Local Stream, also das nahe Umfeld um den Agent schrumpft von $11\times11$
Pixel auf $7\times7$ Pixel. Somit schrumpft gleichzeitig der Action-Space (siehe
\nameref{sub:t_rl_func}) des Agenten von $2\cdot11\cdot11 = 242$ Actions auf
$2\cdot7\cdot7 = 98$ Actions. Das bedeutet für den Agent, dass er sich pro Step
um maximal drei Pixel von seiner Position wegbewegen kann. Diese Bewegung kann
der Agent entweder zeichnend oder nicht zeichnend ausführen (siehe autoref{zeich
vs nicht zeich}).

%todo bild zeichnend vs nicht zeichnend

Falls der Agent die Action zeichnend ausführt, zieht das Programm einen Strich
zwischen der alten und der neuen Position. Mit anderen Worten werden alle Pixel
der Zeichenfläche zwischen den beiden Positionen weiss. Der Strich hat eine
festgelegte Breite von $3$ Pixeln. Am Anfang jeder Episode, also bei jeder neuen
Ziffer, die gezeichnet werden soll, startet der Agent in einer zufälligen
Position im nicht zeichnenden Zustand. Am Anfang jeder Episode ist die
Zeichenfläche leer, also vollkommen Schwarz.

Actions des Agents, die ihn über die vorgegebene Zeichenfläche hinaus
positionieren würden, sind nicht zulässig. Diese Actions können vom Agent nicht
gewählt werden und ihr optimaler Q-Value (siehe autoref{zeich vs nicht zeich})
ist in jedem Fall $0$. Das hat zur Folge, dass nach dem Training die
allermeisten unzulässigen Actions einen Q-Value nahe oder gleich $0$ haben. Das
senkt die Wahrscheinlichkeit, dass der Agent versucht, eine unzulässige Action
auszuführen.

\subsection{Präparierung der Daten und Optimierung}
\label{sub:m_grund_data}
Die Trainingsdaten bestehen aus $36'000$ Bildern von handgeschriebenen Ziffern
aus dem MNIST Datenset (siehe \nameref{chap:t_ml}). Die restlichen Bilder des
MNIST Datensets machen die Testdaten aus. Die Bilder im Datenset sind als Bitmap
dargestellt, wobei jedes Element (jeder Pixel) einen Wert zwischen $0$ und $255$
annimmt. Die Zahl repräsentiert eine Graustufe, wobei $0$ Schwarz ist und $255$
Weiss. Diese Graustufen werden entfernt. Jeder Pixel mit einem Wert über $0$
übernimmt den Wert $1$, wodurch die Bilder nur noch aus Einsen und Nullen
bestehen. Dabei ist $0$ Schwarz und $1$ Weiss (siehe autoref{norm vs. nogray}).
So stimmen die Bilder mit den Zeichnungen, die der Agent produzieren kann,
überein.

%todo Bild normal num vs nogray num

%todo Parameter in Zukunft ändern
Das Grundprogramm trainiert mit $4000$ Bildern, von denen jede Ziffer $400$
Bilder ausmacht. Die restlichen Bilder in den Trainingsdaten sind für mögliche
Erweiterungen aufgehoben. Der Agent zeichnet jedes der $4000$ Bilder ein Mal und
trainiert somit für $4000$ Episodes. Der Agent macht $64$ Steps pro Episode. Er
kann sich also pro Zeichnung $64$ Mal bewegen. Das neuronale Netz passt sich in
jedem vierten Step an, mit einem Batch von $64$ zufällig ausgewählten Steps aus
dem Replay Buffer.

Die Hyperparameter des Grundprogrammes, wie auch der Variationen (siehe
\nameref{chap:m_var}) sind durch den Bayesian Optimization Algorithmus optimiert
(siehe \nameref{sub:t_ml_hyper}). Die Implementierung des Algorithmus in Python
stammt von \cite{fernando_bayesian_2022}. Der Algorithmus ändert sich für
verschiedene Variationen der KI nicht und ist somit Teil des Grundprogrammes,
wobei er zu der optimalen Leistung der KI beiträgt. 

Mit jeder Iteration des Baysian Optimization Algorithmus trainiert das
Reinforcement Learning Modell für eine vom Algorithmus selbst bestimmte Anzahl
Episodes. Die Zielvariable, die durch den Baysian Optimization Algorithmus
maximiert werden soll, wird am Ende jeder Iteration des Trainings in der
Testumgebung berechnet (siehe \nameref{chap:m_auswert_test}). Auf welchem Kriterium
die Zielvariable basiert, ist frei wählbar.



\section{Evaluierung der Leistung}
\label{chap:m_eval}
In diesem Unterkapitel sind die Kriterien definiert, die die Leistung der
künstlichen Intelligenz evaluieren. Mit anderen Worten beschreiben die
Kriterien, wie gut die KI nachzeichnet. Für eine präzise und objektive
Evaluierung sind alle Kriterien durch einen Zahlenwert definiert. Dieser
Zahlenwert geht direkt aus Berechnungen vom Computerprogramm hervor. Die
Kriterien und ihre jeweilige Berechnung sind nachfolgend beschrieben.

\subsection{Erkennbarkeit}
\label{sub:m_eval_rec}
Das Kriterium der Erkennbarkeit beschreibt, ob in der Vorlage das gleiche Motiv
wie in der Zeichnung der künstlichen Intelligenz erkannt wird. Wenn
Beispielsweise in beiden Fällen eine Fünf erkannt wird, hat das Kriterium den
Wert $1$. Wird in der Vorlage eine Fünf erkannt, aber in der Zeichnung eine
Vier, hat das Kriterium den Wert $0$

Welches Motiv in der Zeichnung erkannt wird, ist durch eine zweite künstliche
Intelligenz bestimmt. Diese künstliche Intelligenz beurteilt ein Motiv nur als
erkannt, wenn das zugehörige Neuron im Output des neuronalen Netzen einen Wert
von über $0.9$ hat. Das entspricht mit einer hohen  Wahrscheinlichkeit der
korrekten Beurteilung.

Um die verschiedenene Arten von Strichbildern, die die KI
zeichnen soll, zu erkennen, existieren vortrainierte Machine Learning Modelle. 
Die in dieser Arbeit implementierten vortrainierten Modelle sind in der Tabelle
{...} ersichtlich. Diese Modelle sind mit den selben Daten trainiert, die in der
Testumgebung (siehe \nameref{sub:m_auswert_test}) als Vorlage zum Abzeichnen
dienen.

%todo Tabelle
Tabelle:
Art       |       Entwickler      |     Trainiert mit     |      Genauigkeit 
Zahlen  
Buchstaben
Strichbilder von Objekten

Dieses Kriterium ist in der Fragestellung (siehe \nameref{chap:einleit})
angedeutet. Die Antwort auf die Frage fällt positiv aus, wenn die KI dieses
Kriterium der Erkennbarkeit konsequet erfüllt. Neben der Erkennbarkeit
existieren weitere Kriterien, die andere Aspekte der Leistung der künstlichen
Intelligenz betreffen.

\subsection{Prozentuale Übereinstimmung}
\label{sub:m_eval_proc}
Dieses Kriterium ist durch die die prozentuale Übereinstimmung der weissen
(gezeichneten) Pixel zwischen der Vorlage und der Zeichnung der künstlichen
Intelligenz definiert. Der Wert $K$ dieses Kriteriums zu einem bestimmten
Schritt $t$ berechnet sich aus folgender Formel:
\begin{equation}
  \label{eq:m_reward}
  K(t) = \frac{G(t)}{G_{max}}
\end{equation}
$G_{max}$ entspricht der Anzahl aller weissen Pixeln in der Vorlage. $G(t)$
entspricht der Anzahl der weissen Pixel, die zwischen der Vorlage und der
Zeichenfläche übereinstimmen. Wenn der gleiche Pixel (am gleichen Ort) in der
Vorlage und in der Zeichenfläche weiss ist, erhöht sich diese Anzahl um Eins.
Wenn in der Zeichenfläche ein weisser Pixel gezeichnet ist, der in der Vorlage
schwarz ist, sinkt die Anzahl um Eins. $G(t)$ und somit auch $K(t)$ können
dadurch auch negative Werte annehmen. Der maximale Wert von $K(t)$ ist 1, was
einer Genauigkeit von $100\%$ entspricht (siehe autoref{über Pixel}).

%todo bild übereinstimmende Pixel

\subsection{Geschwindigkeit}
\label{sub:m_eval_speed}
Dieses Kriterium beschreibt, wie schnell die Zeichnung der KI fertig ist. Der
Wert dieses Kriteriums entspricht der Anzahl Steps bis zur Fertigstellung der
Zeichnung. Eine kleinere Anzahl Steps entspricht einer schnelleren Fertigstellung
der Zeichnung und somit einer besseren Leistung nach diesem Kriterium.

Eine Zeichnung gilt als fertig, wenn die prozentuale Übereinstimmung (siehe
\nameref{sub:m_eval_proc}) mindestens $70\%$ beträgt und die Zahl der Definition
entsprechend erkannt wird (siehe \nameref{sub:m_eval_rec}). Wenn die Zeichnung
bis zum Ende der Episode die Bedingungen einer fertigen Zeichnung nicht erfüllt,
hat dieses Kriterium den Wert $64$. Das entspricht der maximalen Anzahl Steps,
die von der KI pro Zeichnung begangen werden.


\section{Variationen}
\label{chap:m_var}
Dieses Kapitel beschreibt Variationen vom Grundprogramm (siehe Grundprogramm).
Bei einigen dieser Variationen handelt es sich um konkrete Implementierungen der
definierten Kriterien in die Reward Function. Die Variationen überschneiden sich
teilweise in den Kriterien, die sie Implementieren. Der Unterschied der
Variationen liegt im Fokus auf die verschiedenen Kriterien. Einige Variationen
sind auch untereinander kombinierbar. Variationen können auch strukturelle
Änderungen an der künsltichen Intelligenz, abgesehen von der Reward Function
haben. Das Ziel dieser strukturellen Änderungen ist eine grundsätzliche
Verbesserung, oder zumindest eine Veränderung der Leistung der künstlichen
Intelligenz. 

%TODO ref
\subsection{Basis Reward Function}
\label{sub:m_var_base}
Die Basis Reward Function ist die einfachste Erweiterung des Grundprogrammes zu
einer funktionierenden künstlichen Intelligenz. Diese Reward Function
implementiert das Kriterium der prozentualen Übereinstimmung (ref proz
Übereinstimmung). Der Reward für eine Aktion berechnet sich aus der Differenz
zwischen der prozentualen Übereinstimmung vor dem Ausführen der Aktion und der
prozentualen Übereinstimmung nach dem Ausführen der Aktion. Somit wird der
Reward $R$ zum Schritt $t$ durch folgende Formel berechnet. (Für $K(t)$: siehe eval)
$$R(t) = K(t) - K(t-1)$$
Der Reward eines Schrittes entspricht dadurch nicht der gesamten prozentualen
Übereinstimmung, sondern lediglich der Veränderung dieser, die durch den Schritt
auslöst. Der addierte Reward aller Schritte enstspricht dem absoluten Wert der
prozentualen Übereinstimmung.


\subsection{Training auf Geschwindigkeit}
\label{sub:m_var_speed}
Der numerische Wert für die Zeit bis zur Fertigstellung der Zeichnung (die
Geschwindigkeit) kann in die Reward-Function integriert werden. Dadurch
trainiert die künstliche Intelligenz auf die kürzeste Zeit bis zur
Fertigstellung. Die Variation verwendet die grundsätzlich die Basis Reward
Function. Die Anpassung davon sieht folgendermassen aus: Am Ende jeder Zeichnung
wird der Reward jedes Schrittes mit einem Faktor $F$ multipliziert. Dieser
Faktor berechnet sich aus folgender Formel:
$$ F = 2 - \frac{S}{S_{max}}$$
$S_{max}$ entspricht der Anzahl Schritte, die der Agent pro Zeichnung (Episode) begeht (siehe Grundprogramm). %TODO: ref setzen.
$S$ Entspricht der Anzahl Schritte zur Fertigstellung der Zeichnung. Der Faktor
nimmt einen Wert zwischen $1$ und $2$ an. Wenn der Agent die Zeichnung bis zum
Ende einer Episode nicht fertigstellen kann, ist $F = 1$. In diesem Fall wird
der Reward also nicht angepasst. Der Agent zeichnet immer $S_{max}$ Schritte pro
Episode. Das verhindert eine ungleichmässige Verteilung der verschiedenen
Episoden im replay-buffer. $S$ wird bis an das Ende der Episode gespeichert,
wenn die Anforderungen für die Fertigstellung einer Zeichnung das erste Mal
erfüllt sind. 
Die Anpassung der Bedingung für eine fertige Zeichnung während dem Training
ermöglicht eine weitere Verbesserung der Geschwindigkeit. Die minimale
prozentuale Übereinstimmung einer fertigen Zeichnung ist als $80\%$ definiert.
Zu beginn des Trainings wird dieser Wert auf $30\%$ heruntergesetzt, und über
das Training hinweg linear bis auf $80\%$ erhöht. Dadurch löst die Reward
Function bereits bei einer unfertigen Zeichnung positive Rewards für die
Geschwindigkeit aus, was den Fokus auf eine maximale Geschwindigkeit weiter
erhöht.


\subsection{Training auf Erkennbarkeit}
\label{sub:m_var_rec}
Das Kriterium der Erkennbarkeit kann, anders als die anderen Kriterien, nur
teilweise in die Reward Function integriert werden. Das Kriterium strebt eine
Erkennbarkeit, unabhängig von der Art der Strichbilder, an (siehe eval
erkennbarkeit). Die künstliche Intelligenz trainiert allerdings nur auf das
Nachzeichnen von Ziffern. Aus diesem Grund trainiert diese Variation nur auf die
Erkennbarkeit von Ziffern, und lässt die anderen Arten von Strichbildern ausser
vor. 

Die Reward Function beinhaltet die künstliche Intelligenz, die handgeschriebene
Ziffern erkennt (siehe eval erkennbarkeit). Diese künstliche Intelligenz
beurteilt in jedem Schritt, welche Ziffern in der Vorlage und der aktuellen
Zeichnung erkannt werden. Wenn die erkannte Zahl in der Vorlage und der
Zeichnung gleich ist, erhält der Agent einen Reward von $0.1$. In diesem Zustand     %TODO Code präzision
funktioniert die Reward-Funktion nicht. Das heisst, der Agent kann den
akkumulierten Reward nicht vergrössern. Zwei Ansätze lösen dieses Problem. Beide
Ansätze sind Teil dieser Variation

Der erste Ansatz ist es, die künstliche Intelligenz, die Ziffern erkennt, erst
ab einer gewissen prozenutalen Übereinstimmung (siehe prozentuale
Übereinstimmung) einzusetzen. Das heisst, dass die korrekte Erkennung erst ab
einer prozentualen Übereinstimmung von $20\%$ einen positiven Reward auslöst.
Diese zusätzliche Bedingung ist notwendig, weil die Einschätzung der Ziffern
durch die künstlichen Intelligenz teilweise für einen menschlichen Betrachter
fragwürdig ist. Zum Beispiel schätzt die Schrifterkennungssoftware eine leere
Zeichenfläche mit einer hohen Wahrscheinlichkeit als eine Eins ein. Das ist in
diesem Fall ein Problem, weil dadurch der Agent einen positiven Reward (eine
Belohnung) für eine leere Zeichenfläche erhält. Das stört das weitere
Lernverhalten, indem es wahrscheinlicher wird, dass der Agent nicht mehr
zeichnet. 

Der zweite Ansatz beinhaltet ebenfalls das Kriterium der prozentualen
Übereinstimmung. Dieses Kriterium ist in dieser Variation gleich wie in der
Basis Reward Function implementiert. Der Unterschied ist, dass der Reward durch
diese Reward Function über das Training hinweg linear kleiner wird. Das Training
startet mit 100\% dieses Rewards und sinkt bis zur letzten Episode des Trainings
auf 10\%. Gleichzeitig nimmt der Reward durch die korrekte Erkennung der Ziffer
linear zu. Die Reward Function ändert sich somit über den Verlauf des Trainings
hinweg. Das hat den Vorteil, dass die künstliche Intelligenz am Anfang des
Trainings durch das Kriterium der prozentualen Übereinstimmung bereits für
kleine Erfolge positive Rewards bekommt. Das Kriterium der Erkennbarkeit
ermöglichst erst bei einer korrekten Erkennung einen Reward. Diese korrekte
Erkennung ist für eine untrainierte künstliche Intelligenz schwer zu erreichen,
wodurch sie nur in seltenen Fällen einen positiven Reward erreichen würde.


\subsection{Physikalische Umgebung}
\label{sub:m_var_phy}

Diese Variation spezialisiert auf kein Kriterium. Stattdessen ist die Umgebung,
in der sich der Agent bewegt, verändert. Auch der Input und der Output des
neuronalen Netzes sind angepasst. Durch diese Veränderungen unterscheidet sich die
Variation vom Grundprogramm. Sie bleibt allerdings mit den anderen
Variationen, die hauptsächlich die Reward Function anpassan, kompatibel.

Die Variation verändert die Umgebung, in der der Agent sich bewegt, in eine
simulierte physikalische Umgebung. Diese Umgebung definiert die physichen
Rahmenbedingungen des Zeichnens neu und bringt diese optimalerweise näher an die
Realität (siehe Diskussion). 

Der Agent hat neu eine Geschwindigkeit, die durch einen Vektor $v$ dargestellt
ist. Die Geschwindigkeit beschreibt, um wie viele Pixel und in welche Richtung
sich der Agent pro Schritt bewegt.
Die folgende Formel beschreibt, wie sich die Position des Agenten von Schritt
$t$ bis zum nächsten Schritt $t+1$ ändert:
$$p_{t+1} = p_t + v_t$$
$p_n$ beschreibt die Postition des Agenten zu Schritt $n$ und $v_n$ beschreibt
die Geschwindigkeit des Agenten zu Schritt $n$. Die Position rundet in jedem
Schritt auf ganze Zahlen. Das kommt daher, dass die Geschwindigkeit auch
Dezimalzahlen annehmen kann, aber die Positionnur durch ganze Zahlen dargestellt
ist.

Zur Geschwindigkeit des Agenten wird in jedem Schritt ein Beschleunigungsvektor
addiert. Jede Action, die der Agent wählen kann, entspricht einem anderen
Beschleunigungsvektor. Der Action Space besteht neu aus 42 Aktionen. Davon
entsprechen 21 Aktionen Beschleunigungsvektoren im zeichnenden Zustand. Die
anderen 21 Aktionen entsprechen den selben Vektoren im nicht zeichnenden
Zustand. Die 21 verschiedenen Beschleunigungsvektoren sehen folgendermassen aus:
(siehe Abbildung action Kreis) Ein Vektor entspricht dem Nullvektor. Dieser
verändert die Geschwindigkeit des Agenten nicht. $8$ Vektoren sind um den
Agenten herum mit einer Länge von $0.8$ Pixeln in gleichmässigem Absant von
einander angeordnet. Zusammen bilden diese Vektoren einen Kreis um den Agent die
restlichen 12 Vektoren sind in einem grösseren Kreis gleichmässig angeordnet.
Die Vektoren haben dabei eine Länge von $1.2$ Pixeln. Der Betrag der
Geschwindigkeit des Agenten wird, unabhängig von der gewählten Aktion, in jedem
Schritt um $0.4$ Pixel verringert. Das Simuliert eine Reibungskraft, die auf den
Agenten einwirkt.
Mit dem gewählten beschleunigungsvektor $a_t$ berechnet sich die Geschwindigkeit
im nächsten Schritt $t+1$ aus dem aktuellen Schritt $t$ durch folgende Formel:
$$v_{t+1} = v_t + a_t - 0.4$$

Die Änderungen in der Umgebung erfordern weitere Anpassungen im neuronalen Netz.
Ohne diese Anpassungen lernt die künstliche Intelligenz nicht. Das Problem ist,
dass die aktuelle Geschwindigkeit kein Teil der Observation ist. Die
Entscheidungen des Agenten berücksichtigen dadurch dessen Geschwindigkeit nicht.
Die Verschiebung von dem local patch bietet eine Lösung für dieses Problem. Im
Grundprogramm entspricht der Mittelpunkt von dem Local Patch genau der Position
des Agents. Neu befindet sich der Mittelpunkt dort, wo sich der Agent laut
seiner aktuellen Geschwindigkeit im nächsten Schritt befindet (Die tatsächliche
neue Position des Agents wird durch die Action seiner Wahl bestimmt. Wie im
Grundprogramm gibt der local patch den Bereich an, in dem sich der Agent im
nächsten Schritt befinden wird). Diese Verschiebung vom Local Patch ist eine
implizite Angabe der Geschwindigkeit des Agents. Der Local Patch hat neu eine
Grösse von $5\times5$ Pixeln an der Stelle von $7\times7$ Pixeln.

%Bild Verschiebung des local patch

Ein weiteres Problem ist, dass der Agent sich durch seine Geschwindigkeit aus
den vorgegebenen Grenzen der Zeichenfläche begeben kann. Im Grundprogramm  
(siehe) kann der Agent Aktionen, die ihn in diese Unzulässigen Positionen  %TODO ref
bewegen würden, nicht auswählen. Wenn die Geschwindigkeit zu hoch wird, kann der
Agent allerdings gar keine Aktionen mehr wählen, die ihn innerhalb der Grenzen der
Zeichenflächen halten würden. Wenn der Agent durch zu hohe Geschwindigkeit über
die Grenze hinausgeht, wird seine Geschwindigkeit auf den Nullvektor
zurückgesetzt und die Reward Function löst einen negativen Reward von $-0.05$
aus. Dieser negative Reward soll zu hohe Geschwindigkeiten an gewissen
Positionen auf der zeichenfläche vermeiden.


\section{Auswertung}
\label{chap:m_auswert}
Die Auswertung der Daten über die Leistung der künstlichen Intelligenz liefert
das Resultat der Methode. Die Auswertung berechnet den Zahlenwert der
definierten Kriterien für verschiedene Variationen der künstlichen Intelligenz.

Die Variationen werden auf ihre Leistung für drei verschiedene Datensets
überprüft. Die drei Datensets beinhalten verschiedene Arten von handgemachten
Strichbildern Das erste Datenset, MNIST (mnist ref),  beeinhaltet Ziffern, die
in den Trainingsdaten nicht vorkommen. Das zweite Datenset, EMNIST letters
(emnist ref), beeinhaltet die 26 Kleinbuchstaben des Alphabets. Das dritte
Datenset, QuickDraw (quickdraw ref), beeinhaltet Zeichnungen von insgesamt 345
verschiedenen Motiven. die Variationen werden allerdings nur auf das
Nachzeichnen von zehn Motiven überprüft. Die zehn Motive sind: `Amboss',
`Apfel', `Besen', `Eimer', `Bulldozer', `Uhr', `Wolke', `Computer', `Auge' und
`Blume'. Die Bilder in den drei Datensets sind gleich verarbeitet wie die
Trainingsdaten (siehe Prep daten). 

Folgende Variationen der künstlichen Intelligenz werden ausgewertet:

\begin{itemize}
  \item Grundumgebung + Basis Reward Functin
  \item Grundumgebung + Erkennbarkeit
  \item Grundumgebung + Geschwindigkeit
  \item Grundumgebung + Erkennbarkeit + Geschwindigkeit
  \item physikalische Umgebung + Basis Reward Functin
  \item physikalische Umgebung + Erkennbarkeit
  \item physikalische Umgebung + Geschwindigkeit
  \item physikalische Umgebung + Erkennbarkeit + Geschwindigkeit
\end{itemize}

%Bild von Quickdraw

\subsection{Testumgebung}
\label{sub:m_auswert_test}

Die Leistungen der verschiedenen Variationen der künstlichen Intelligenz werden
in einem Test, in einer Testumgebung ausgewertet. Die Testumgebung unterscheidet
sich kaum von der Trainingsumgebung. Es gibt im Wesentlichen drei Unterschiede.
Erstens trainiert die künstliche Intelligenz in der Testumgebung nicht. Die
Testumgebung übernimmt eine trainierte künstliche Intelligenz und verändert
diese während dem Test nicht. Zweitens wählt der Agent in keinem Fall mehr eine
zufälligen Aktion. Stattdessen wählt er immer die Aktion mit dem höchsten
Q-Value (gleichbedeutend mit $\epsilon = 0$). Der Dritte Unterschied liegt in
den Strichbildern, die für die künstliche Intelligenz als Vorlage dienen. Im
Test zeichnet das Computerprogramm $1040$ Bilder aus einem der drei zur
Verfügung stehenden Datensets. 

Am Ende jeder Episode (d.h jeder Zeichnung) wird der Zahlenwert für die
verschiedenen Kriterien nach ihrer Definition ausgewertet und gespeichert. Die
KI zeichnet auch in der Testumgebung für $64$ Steps. Wenn eine Zeichnung der
Definition entsprechend früher fertig ist (siehe \nameref{sub:m_eval_speed}),
wird die Anzahl Steps zu diesem Zeitpunkt als Wert des Kriteriums der
Geschwindigkeit gespeichert und die KI zeichnet weiter. Der Durchschnitt aller
gespeicherten Werte eines Kriteriums entspricht der Leistung der getesteten
Variation in diesem Kriterium. Das Kriterium der Erkennbarkeit verwendet zur
Auswertung je nach dem verwendeten Datenset im Test, dasjenige vortrainierte
Modell, das auf dem selben Datenset trainiert ist. Da das Kriterium der
Erkennbarkeit jeweils den Wert $0$ oder $1$ hat, ergibt der Durchschnitt aus
allen Werten eine prozentuale Angabe (in Dezimalform) darüber, in wie vielen
Fällen das richtige Motiv erkannt wird.



