\chapter{Methode}\label{chap:m}
Die Methode dieser Untersuchung besteht darin, die in der Fragestellung
beschriebene künstliche Intelligenz (KI) zu entwickeln und dessen Leistung
auszuwerten. Die Diskussion dieser Resultate führt schlussendlich zu einer
Antwort auf die Fragestellung. Die Entwicklung der KI besteht aus zwei Teilen.
Der eine Teil umfasst die Definition der Kriterien, nach denen die Leistung der
KI evaluiert wird (siehe \nameref{chap:m_eval}). Der andere Teil umfasst die
Entwicklung der KI (siehe \nameref{chap:m_grund}), zusammen mit verschiedenen
Variationen (siehe \nameref{chap:m_var}). Die Variationen haben jeweils einen
unterschiedlichen Fokus auf die definierten Kriterien. Die Auswertung (siehe
\nameref{chap:m_auswert}) bezieht sich ebenfalls auf die definierten Kriterien.
Die Leistung der KI wird dabei in einer Testumgebung für das Zeichnen von
verschiedenen Arten von Strichbildern erfasst.

\section{Grundprogramm}\label{chap:m_grund}
Die KI ist abhängig von den Kriterien, die dessen Leistung definieren (siehe
\nameref{chap:m_eval}). Mit anderen Worten trainiert die KI auf diese Kriterien.
Das Ziel des Grundprogrammes ist, die allgemeine Trainingsumgebung für die KI
bereitzustellen. Dieses Grundprogramm ist unabhängig von einem spezifischen
Kriterium und kann stattdessen auf ein ausgewähltes Kriterium trainiert werden.
Reinforcement Learning Modelle mit einer undefinierten Reward Function (siehe
\nameref{sub:t_rl_func}) stellen diese Eigenschaften bereit. Eine Reward
function, basierend auf einem spezifischen Kriterium, ermöglicht das Training
auf dieses Kriterium. Das Grundprogramm ist in Python unter der Verwendung des
Keras Frameworks implementiert (siehe \nameref{chap:t_ml}). 

\subsection{Doodle-SDQ als Basis}\label{sub:m_grund_dood}
Das Reinforcement Learning Modell des Grundprogrammes basiert auf Doodle-SDQ.
(siehe \nameref{sub:t_ver_dood}) Von Doodle-SDQ ist das neuronale Netz, bezogen
auf die Form des Inputs, des Outputs und den Hidden Layers, zu grossem Teil
übernommen. Die relevanten Anpassungen zwischen Doodle-SDQ und dem Grundprogramm
dieser Arbeit sind nachfolgend erläutert.

Bei der Umgebung handelt es sich, wie bei Doodle-SDQ, um eine Zeichenfläche,
worauf sich der Agent frei bewegen kann. Die Ziffern, die während dem Training
nachgezeichnet werden sollen, stammen aus dem MNIST Datenset (siehe
\nameref{chap:t_ml}) und haben somit eine Grösse von $28\times28$ Pixeln. Die
Fläche, worauf sich der Agent bewegen   
kann, hat somit auch eine Grösse von $28\times28$ Pixeln. Der global Stream
(siehe \nameref{sub:t_ver_dood}) des Inputs in das Neuronale Netz ändert sich
bis auf die neue Grösse der Bilder nicht. Die Pixel der Bilder, wie auch die
Zeichenfläche, nehmen den Wert von einem Bit an. Eine Null repräsentiert einen
schwarzen (nicht gezeichneten) Pixel an dieser Stelle im Bild und eine Eins
einen weissen (gezeichneten) Pixel. Die genaue Architektur des neuronalen Netzes
ist im Schema autoref{architecture} angegeben. Jeder Block in der Abbildung
repräsentiert Eine Layer des neuronalen Netzes, wobei die Form des Inputs und
die Form des Outputs von jeder Layer angegeben ist. 

%TODO Bild architecture

Der Local Stream, also das nahe Umfeld um den Agent schrumpft von $11\times11$
Pixel auf $7\times7$ Pixel. Somit schrumpft gleichzeitig der Action-Space (siehe
\nameref{sub:t_rl_func}) des Agenten von $2\cdot11\cdot11 = 242$ Actions auf
$2\cdot7\cdot7 = 98$ Actions. Das bedeutet für den Agent, dass er sich pro Step
um maximal drei Pixel von seiner Position wegbewegen kann. Diese Bewegung kann
der Agent entweder zeichnend oder nicht zeichnend ausführen (siehe autoref{zeich
vs nicht zeich}).

%todo bild zeichnend vs nicht zeichnend

Falls der Agent die Action zeichnend ausführt, zieht das Programm einen Strich
zwischen der alten und der neuen Position. Mit anderen Worten werden alle Pixel
der Zeichenfläche zwischen den beiden Positionen weiss. Der Strich hat eine
festgelegte Breite von $3$ Pixeln. Am Anfang jeder Episode, also bei jeder neuen
Ziffer, die gezeichnet werden soll, startet der Agent in einer zufälligen
Position im nicht zeichnenden Zustand. Am Anfang jeder Episode ist die
Zeichenfläche leer, also vollkommen Schwarz.

Actions des Agents, die ihn über die vorgegebene Zeichenfläche hinaus
positionieren würden, sind nicht zulässig. Diese Actions können vom Agent nicht
gewählt werden und ihr optimaler Q-Value (siehe autoref{zeich vs nicht zeich})
ist in jedem Fall $0$. Das hat zur Folge, dass nach dem Training die
allermeisten unzulässigen Actions einen Q-Value nahe oder gleich $0$ haben. Das
senkt die Wahrscheinlichkeit, dass der Agent versucht, eine unzulässige Action
auszuführen.

\subsection{Präparierung der Daten und Optimierung}\label{sub:m_grund_data}
Die Trainingsdaten bestehen aus $36'000$ Bildern von handgeschriebenen Ziffern
aus dem MNIST Datenset (siehe \nameref{chap:t_ml}). Die restlichen Bilder des
MNIST Datensets machen die Testdaten aus. Die Bilder im Datenset sind als Bitmap
dargestellt, wobei jedes Element (jeder Pixel) einen Wert zwischen $0$ und $255$
annimmt. Die Zahl repräsentiert eine Graustufe, wobei $0$ Schwarz ist und $255$
Weiss. Diese Graustufen werden entfernt. Jeder Pixel mit einem Wert über $0$
übernimmt den Wert $1$, wodurch die Bilder nur noch aus Einsen und Nullen
bestehen. Dabei ist $0$ Schwarz und $1$ Weiss (siehe autoref{norm vs.nogray}).
So stimmen die Bilder mit den Zeichnungen, die der Agent produzieren kann,
überein.

%todo Bild normal num vs nogray num

%todo Parameter in Zukunft ändern
Das Grundprogramm trainiert mit $4000$ Bildern, von denen jede Ziffer $400$
Bilder ausmacht. Die restlichen Bilder in den Trainingsdaten sind für mögliche  
Erweiterungen aufgehoben. Der Agent zeichnet jedes der $4000$ Bilder ein Mal und
trainiert somit für $4000$ Episodes. Der Agent macht $64$ Steps pro Episode. Er
kann sich also pro Zeichnung $64$ Mal bewegen. Das neuronale Netz passt sich in
jedem vierten Step an, mit einem Batch von $64$ zufällig ausgewählten Steps aus
dem Replay Buffer.

Die Hyperparameter des Grundprogrammes, wie auch der Variationen (siehe
\nameref{chap:m_var}) sind durch den Bayesian Optimization Algorithmus optimiert
(siehe \nameref{sub:t_ml_hyper}). Die Implementierung des Algorithmus in Python
stammt von \cite{fernando_nogueira_bayesian_2014}. Der Algorithmus ändert sich
für verschiedene Variationen der KI nicht und ist somit Teil des
Grundprogrammes, wobei er zu der optimalen Leistung der KI beiträgt. 

Mit jeder Iteration des Baysian Optimization Algorithmus trainiert das
Reinforcement Learning Modell für eine vom Algorithmus selbst bestimmte Anzahl
Episodes. Die Zielvariable, die durch den Baysian Optimization Algorithmus
maximiert werden soll, wird am Ende jeder Iteration des Trainings in der
Testumgebung berechnet (siehe \nameref{sub:m_auswert_test}). Auf welchem
Kriterium die Zielvariable basiert, ist frei wählbar.



\section{Evaluierung der Leistung}\label{chap:m_eval}
In diesem Unterkapitel sind die Kriterien definiert, die die Leistung der
künstlichen Intelligenz evaluieren. Mit anderen Worten beschreiben die
Kriterien, wie gut die KI nachzeichnet. Für eine präzise und objektive
Evaluierung sind alle Kriterien durch einen Zahlenwert definiert. Dieser
Zahlenwert geht direkt aus Berechnungen vom Computerprogramm hervor. Die
Kriterien und ihre jeweilige Berechnung sind nachfolgend beschrieben.

\subsection{Erkennbarkeit}\label{sub:m_eval_rec}
Das Kriterium der Erkennbarkeit beschreibt, ob in der Vorlage das gleiche Motiv
wie in der Zeichnung der künstlichen Intelligenz erkannt wird. Wenn
Beispielsweise in beiden Fällen eine Fünf erkannt wird, hat das Kriterium den
Wert $1$. Wird in der Vorlage eine Fünf erkannt, aber in der Zeichnung eine
Vier, hat das Kriterium den Wert $0$

Welches Motiv in der Zeichnung erkannt wird, ist durch eine zweite künstliche
Intelligenz bestimmt (siehe \nameref{sub:t_ml_func}). Diese zweite KI beurteilt
ein Motiv nur als erkannt, wenn das zugehörige Neuron im Output des neuronalen
Netzen einen Wert von über $0.75$ hat. Das entspricht mit einer hohen
Wahrscheinlichkeit der korrekten Beurteilung.

Um die verschiedenene Arten von Strichbildern, die die KI zeichnen soll, zu
erkennen, existieren vortrainierte Machine Learning Modelle. Die in dieser
Arbeit implementierten vortrainierten Modelle sind in der Tabelle {...}
ersichtlich. Diese Modelle sind mit den selben Daten trainiert, die in der
Testumgebung (siehe \nameref{sub:m_auswert_test}) als Vorlage zum Abzeichnen
dienen.

%todo Tabelle
\label{tab:Datasets}
\begin{table}[!ht]
  \centering
  \begin{tabular}{|l|l|l|l|}
  \hline
      Art & Entwickler & Trainiert mit \\ \hline
      Zahlen & \cite{mazzia__2022} & MNIST \\ \hline
      Buchstaben & \cite{mor_emnist_2022} & EMNIST Letters \\ \hline
      \makecell{Strichbilder\\von Objekten} & \cite{lam_linus_keras_2022} & Auswahl aus QuickDraw \\ \hline
  \end{tabular}
\end{table}

Dieses Kriterium ist in der Fragestellung (siehe \nameref{chap:einleit})
angedeutet. Die Antwort auf die Frage fällt positiv aus, wenn die KI dieses
Kriterium der Erkennbarkeit konsequet erfüllt. Neben der Erkennbarkeit
existieren weitere Kriterien, die andere Aspekte der Leistung der künstlichen
Intelligenz betreffen.

\subsection{Prozentuale Übereinstimmung}
\label{sub:m_eval_proc}
Dieses Kriterium ist durch die die prozentuale Übereinstimmung der weissen
(gezeichneten) Pixel zwischen der Vorlage und der Zeichnung der künstlichen
Intelligenz definiert. Der Wert $K$ dieses Kriteriums zu einem bestimmten Step
$t$ berechnet sich aus folgender Formel:
\[ K(t) = \frac{G(t)}{G_{\max}} \]
$G_{\max}$ entspricht der Anzahl aller weissen Pixeln in der Vorlage. $G(t)$
entspricht der Anzahl der weissen Pixel, die zwischen der Vorlage und der
Zeichenfläche übereinstimmen. Wenn der gleiche Pixel (am gleichen Ort) in der
Vorlage und in der Zeichenfläche weiss ist, erhöht sich diese Anzahl um Eins.
Wenn in der Zeichenfläche ein weisser Pixel gezeichnet ist, der in der Vorlage
schwarz ist, sinkt die Anzahl um Eins. $G(t)$ und somit auch $K(t)$ können
dadurch auch negative Werte annehmen. Der maximale Wert von $K(t)$ ist 1, was
einer Genauigkeit von $100\%$ entspricht (siehe autoref{über Pixel}).

%todo bild übereinstimmende Pixel

\subsection{Geschwindigkeit}\label{sub:m_eval_speed}
Dieses Kriterium beschreibt, wie schnell die Zeichnung der KI fertig ist. Der
Wert dieses Kriteriums entspricht der Anzahl Steps bis zur Fertigstellung der
Zeichnung. Eine kleinere Anzahl Steps entspricht einer schnelleren Fertigstellung
der Zeichnung und somit einer besseren Leistung nach diesem Kriterium.

Eine Zeichnung gilt als fertig, wenn die prozentuale Übereinstimmung (siehe
\nameref{sub:m_eval_proc}) mindestens $70\%$ beträgt und die Zahl der Definition
entsprechend erkannt wird (siehe \nameref{sub:m_eval_rec}). Wenn die Zeichnung
bis zum Ende der Episode die Bedingungen einer fertigen Zeichnung nicht erfüllt,
hat dieses Kriterium den Wert $64$. Das entspricht der maximalen Anzahl Steps,
die von der KI pro Zeichnung begangen werden.


\section{Variationen}\label{chap:m_var}
Dieses Kapitel beschreibt verschiedene Variationen ausgehend vom Grundprogramm
(siehe \nameref{chap:m_grund}). Bei einigen dieser Variationen handelt es sich
um konkrete Implementierungen der definierten Kriterien in die Reward Function
(siehe \nameref{sub:t_rl_func}). Die Reward Function kann dabei auch auf
mehreren Kriterien basieren. Der Unterschied zwischen den Variationen liegt im
Fokus auf die Kriterien. Einige Variationen sind untereinander kombinierbar.
Andere Variationen führen Strukturelle Änderungen an der KI ein, die über die
Reward Function hinaus gehen. Das Ziel der strukturellen Änderungen ist eine
grundsätzliche Verbesserung, oder zumindest eine Anpassung des Verhaltens und
der Leistung der künstlichen Intelligenz. 

\subsection{Basis Reward-Function}\label{sub:m_var_base}
Die Basis Reward Function ist die einfachste Erweiterung des Grundprogrammes
(siehe \nameref{chap:m_grund}) zu einer funktionierenden künstlichen
Intelligenz. Diese Reward Function implementiert das Kriterium der prozentualen
Übereinstimmung (siehe \nameref{sub:m_eval_proc}). Der Reward für eine Action
berechnet sich aus der Differenz zwischen der prozentualen Übereinstimmung vor
dem Ausführen der Action, und und der prozentualen Übereinstimmung nach dem
Ausführen der Action (also $K(t-1)$ und $K(t)$). Somit wird der Reward $R$ zum
Step $t$ durch folgende Formel berechnet. 
\[ R(t) = K(t) - K(t-1) \]
Der Reward eines Steps entspricht somit nicht der gesamten prozentualen
Übereinstimmung zu diesem Step. Stattdessen Entspricht der Reward der
Veränderung der prozentualen Übereinstimmung, ausgelöst durch die Action zu diesem
Step. Der addierte Reward aller Steps enstspricht dem absoluten Wert der
prozentualen Übereinstimmung.


\subsection{Training auf Geschwindigkeit}\label{sub:m_var_speed}
Der numerische Wert für die Zeit bis zur Fertigstellung der Zeichnung (siehe
\nameref{sub:m_eval_speed}) kann in die Reward-Function integriert werden.
Dadurch trainiert die künstliche Intelligenz auf eine minimale Zeit bis zur
Fertigstellung. Die Variation verwendet die grundsätzlich die Basis
Reward-Function (siehe \nameref{sub:m_var_base}). Die Anpassung davon sieht
folgendermassen aus: Am Ende jeder Zeichnung wird der Reward jedes Steps mit
einem Faktor $f$ multipliziert. Dieser Faktor berechnet sich aus folgender
Formel:
\[ f = 2 - \frac{S}{S_{\max}} \]
$S_{\max}$ entspricht der Anzahl Steps, die der Agent pro Zeichnung (Episode)
begeht (siehe \nameref{sub:m_grund_data}). $S$ Entspricht der Anzahl Steps
bis zur Fertigstellung der Zeichnung. Der Faktor nimmt einen Wert zwischen $1$
und $2$ an. Ein grösserer Faktor $f$ entspricht einer schnellen Fertigstellung
und deswegen einem hohen Reward. Wenn der Agent die Zeichnung bis zum Ende einer
Episode nicht fertigstellen, ist $f = 1$ (siehe \nameref{sub:m_eval_speed}). In
diesem Fall unterscheidet sich die Reward-Function nicht von der Basis
Reward-Function. Wenn die Zeichnung früher fertiggestellt wird, zeichnet der
Agent trotzdem $S_{\max}$ Steps. Das verhindert eine ungleichmässige Verteilung
zwischen verschiedenen Episodes im Replay-Buffer (siehe \nameref{sub:t_rl_func}).
In diesem Fall wird $S$ nur in dem Step gespeichert, in dem die Zeichnung zum
ersten Mal die Bedingung einer Fertigstellung erfüllt.   

Der Fokus des Trainings auf eine maximale Geschwindigkeit erfährt einen weiteren
anstieg, durch eine Anpassung der Bedingung für eine fertige Zeichnung während
dem Training. Die minimale prozentuale Übereinstimmung einer fertigen Zeichnung
ist als $75\%$ definiert. Zu beginn des Trainings wird dieser Wert auf $25\%$  %Todo code präzision
heruntergesetzt, und über das Training hinweg linear bis auf $75\%$ erhöht.
Dadurch löst die Reward Function bei einer unfertigen Zeichnung bereits positive
Rewards für die Geschwindigkeit aus.


\subsection{Training auf Erkennbarkeit}
\label{sub:m_var_rec}
Das Kriterium der Erkennbarkeit kann, anders als die anderen Kriterien, nur
teilweise in die Reward Function integriert werden. Das Kriterium strebt eine
Erkennbarkeit, unabhängig von der Art der Strichbilder, an (siehe
\nameref{sub:m_eval_rec}). Die künstliche Intelligenz trainiert allerdings nur
auf das Nachzeichnen von Ziffern. Aus diesem Grund trainiert diese Variation nur
auf die Erkennbarkeit von Ziffern, und lässt die anderen Arten von Strichbildern
ausser vor. 

Die Reward-Function (siehe \nameref{sub:t_rl_func}) dieser Variation beinhaltet
eine zweite KI, die handgeschriebene Ziffern erkennt (siehe
\autoref{tab:Datasets}). Diese zweite KI beurteilt in jedem Step, welche
Ziffern sie in der Vorlage und in der aktuellen  %TODO Code präzision
Zeichnung erkennt. Wenn die erkannte Zahl in der Vorlage und der Zeichnung
gleich ist, erhält der Agent einen Reward von $0.1$. In diesem Zustand
funktioniert die Reward-Function allerdings nicht. Der Agent kann den
akkumulierten Reward nicht maximieren. Zwei Ansätze gehen auf dieses Problem
ein. Beide Ansätze sind Teil dieser Variation.

Der erste Ansatz schlägt vor, die zweite KI erst ab einer gewissen prozentualen
Übereinstimmung (siehe \nameref{sub:m_eval_proc}) einzusetzen. In diesem Fall
löst die korrekte Erkennung erst ab einer prozentualen Übereinstimmung von
$20\%$ einen positiven Reward aus. Diese zusätzliche Bedingung ist notwendig,
weil die Beurteilungen der zweiten KI teilweise für einen menschlichen
Betrachter fragwürdig sind. Zum Beispiel schätzt die zweite KI eine leere
Zeichenfläche mit einer hohen Wahrscheinlichkeit als eine Eins (siehe
autoref{mnist rec}) ein. Das ist ein Problem, weil dadurch der Agent einen
positiven Reward für eine leere Zeichenfläche erhält. Das stört das weitere
Lernverhalten, weil es die Wahrscheinlichkeit erhöht, dass die KI nicht mehr
zeichnet.

%todo Bild mnist rec

Der zweite Ansatz implementiert neben der Reward-Function der Erkennbarkeit
erneut die Basis Reward-Function (siehe \nameref{sub:m_var_base}). Die Relevanz
der beiden Reward-Functions ändert sich allerdings über das Training hinweg. Die
Rewards werden in jedem Step mit einem bestimmten Faktor multipliziert. Zu
Beginn des Trainings ist der Faktor für den Reward der Basis Reward-Function
$f_b = 1$ und der Faktor für den Reward basierend auf der Erkennbarkeit $f_e =
0$. Vom Start ausgehend sinkt $f_b$ linear und $f_e$ steigt linear. Ab einem
gewissen Punkt bleiben beide Faktoren stehen (siehe autoref{decrementor graph}).
Blieben die Faktoren ab diesem Punkt nicht konstant, würde die Variation,
gestützt auf Beobachtungen, an Stabilität der Leistung verlieren.

%todo Bild Decrementor

Das Zusammenspiel der beiden Reward-Functions hat den Vorteil, dass die
künstliche Intelligenz zu Beginn des Trainings durch die Basis Reward-Function
für kleine Erfolge positive Rewards erzielt. Die Reward-Function der
Erkennbarkeit ermöglicht das nicht, da sie erst für eine korrekte Erkennung
einen Reward auslöst. Eine korrekte Erkennung ist für eine untrainierte KI
schwer zu erreichen. Deswegen muss die KI durch die Basis Reward-Function
gewissermassen vortrainiert werden, um schlussendlich von der Reward-Function
der Erkennbarkeit zu profitieren


\subsection{Physikalische Umgebung}\label{sub:m_var_phy}
Diese Variation spezialisiert sich auf kein Kriterium. Stattdessen verändert
sich die Umgebung, in der sich der Agent bewegt (siehe \nameref{sub:t_rl_func}).
Auch der Input und der Output des neuronalen Netzes sind angepasst. Durch diese
Veränderungen unterscheidet sich die Variation vom Grundprogramm. Sie bleibt
allerdings mit den anderen Variationen (siehe \nameref{sub:m_var_rec} und
\nameref{sub:m_var_speed}) kompatibel, Da diese ausschliesslich die
Reward-Function anpassen.

Die Variation ergänzt die Umgebung durch physikalische Simulationen. Diese
physikalische Umgebung definiert die physichen Rahmenbedingungen des Zeichnens
neu, mit dem Ziel, diese näher an die Realität zu bringen (siehe
\nameref{sub:d_frage_unter}). 

Der Agent hat neu eine Geschwindigkeit, die durch einen Vektor $\vec{v}$
dargestellt ist. Die Geschwindigkeit beschreibt, um wie viele Pixel und in
welche Richtung sich der Agent pro Step bewegt. Die folgende Formel beschreibt,
wie sich die Position des Agenten vom Step $t$ bis zum nächsten Step $t+1$
ändert:
\[ \vec{p}(t+1) = \vec{p}(t) + \vec{v}(t) \] 
$\vec{p}(t)$ beschreibt die Position des Agents als einen Ortsvektor auf der
Zeichenfläche zum Step $t$ und $\vec{v}(t)$ beschreibt die Geschwindigkeit des
Agenten zum Step $t$. Die Position rundet in jedem Step auf ganze Zahlen. Das
kommt daher, dass die Geschwindigkeit auch Dezimalzahlen annehmen kann, aber die
Position nur durch ganze Zahlen dargestellt ist.

Zur Geschwindigkeit des Agent wird in jedem Step ein Beschleunigungsvektor
addiert. Jede Action, die der Agent wählen kann, entspricht einem anderen
Beschleunigungsvektor. Der Action-Space (siehe \nameref{sub:t_rl_func}) besteht
neu aus $42$ Actions. $21$ der $42$ Actions entsprechen Beschleunigungsvektoren
im zeichnenden Zustand. Die anderen $21$ Actions entsprechen den selben Vektoren
im nicht zeichnenden Zustand. Die 21 verschiedenen Beschleunigungsvektoren haben
folgende Form: Ein Vektor entspricht dem Nullvektor. Dieser verändert die
Geschwindigkeit des Agents nicht. $8$ Vektoren sind um den Agent herum mit
einer Länge von $0.9$ Pixeln in gleichmässigem Abstand von einander angeordnet. %todo code präzision
Zusammen bilden diese Vektoren einen Kreis um den Agent. Die restlichen 12
Vektoren sind in einem grösseren Kreis gleichmässig angeordnet. Die Vektoren
haben dabei eine Länge von $1.8$ Pixeln (siehe autoref{actionspace}).  Mit dem
gewählten Beschleunigungsvektor $\vec{a}(t)$ berechnet sich die Geschwindigkeit
im nächsten Step $t+1$ aus dem aktuellen Step $t$ durch folgende Formel:
\[ \vec{v}(t+1) = \vec{v}(t) + \vec{a}(t) \] 
Der Betrag der Geschwindigkeit $\vec{v}(t+1)$ des Agents wird, unabhängig von
der gewählten Action, in jedem Step um $0.2$ Pixel verringert. Das simuliert eine
Reibungskraft, die auf den Agent einwirkt.

%todo bild actionspace

Die Veränderung in der Umgebung erfordert weitere Anpassungen im neuronalen Netz
(siehe \nameref{sub:t_ml_nn}). Ohne diese Anpassungen kann die KI den
akkumulierten Reward nicht maximieren. Das Problem ist, dass die aktuelle
Geschwindigkeit des Agents kein Teil der Observation (siehe
\nameref{sub:t_rl_func}) ist. Der Agent berücksichtigt deswegen seine
Geschwindigkeit nicht in seinen Entscheidungen. Die Lösung dieses Problems
bietet eine Verschiebung des Local image patch (siehe \nameref{sub:t_ver_dood}).
Im Grundprogramm entspricht der Mittelpunkt des Local image patchs genau der
Position des Agents. Neu befindet sich der Mittelpunkt dort, wo sich der Agent
laut seiner aktuellen Geschwindigkeit im nächsten Schritt befinden wird. Durch
diese Verschiebung des Local image Patches erhält der Agent Informationen über
seine Geschwindigkeit, ohne dessen numerischen Wert zu kennen. Wie im
Grundprogramm gibt der Local image patch den gesamten Bereich an, in dem sich
der Agent im nächsten Schritt befinden kann. Die tatsächliche neue Position des
Agents wird durch die Action seiner Wahl bestimmt. Die Grösse des Local image
patches schrumpft von $7\times7$ Pixeln auf $5\times5$ Pixel, da alle möglichen
Positionen des Agents nach einem Step auf einem $5\times5$ Feld Platz haben
(siehe autoref{local patch}). 

%todo bild local patch verschiebung

Ein weiteres Problem ist, dass der Agent sich durch seine Geschwindigkeit aus
den vorgegebenen Grenzen der Zeichenfläche begeben kann. Im Grundprogramm  
(siehe \nameref{sub:m_grund_dood}) kann der Agent Actions, die ihn in eine
unzulässige Position bewegen würden, nicht auswählen. Wenn allerdings in der
phyiskalischen Umgebung die Geschwindigkeit des Agents zu
hoch ist, kann dieser keine Actions mehr wählen, die ihn innerhalb der Grenzen
der Zeichenflächen hielten. Wenn der Agent durch zu hohe Geschwindigkeit
über die Grenze hinausgeht, wird seine Geschwindigkeit auf den Nullvektor
zurückgesetzt und die Reward Function löst einen negativen Reward von $-0.05$
aus. Der negative Reward soll die Häufigkeit dieser Vorfälle vermindern


\section{Auswertung}\label{chap:m_auswert}
Die Auswertung der Daten über die Leistung der künstlichen Intelligenz liefert
das Resultat der Methode. Die Auswertung berechnet den Zahlenwert der
definierten Kriterien (siehe \nameref{chap:m_eval}) für verschiedene Variationen
der KI.

Die Variationen werden auf ihre Leistung für drei verschiedene Datensets
überprüft. Die drei Datensets beinhalten verschiedene Arten von handgemachten
Strichbildern. (siehe \autoref{tab:Datasets}) Das erste Datenset, MNIST,
beeinhaltet Ziffern, die in den Trainingsdaten nicht vorkommen. Das zweite
Datenset, EMNIST Letters, beeinhaltet die 26 Kleinbuchstaben des Alphabets. Das
dritte Datenset, QuickDraw, beeinhaltet Zeichnungen von insgesamt 345
verschiedenen Motiven. Die KI wird allerdings nur auf das Nachzeichnen von zehn
Motiven überprüft. Die zehn Motive sind: `Amboss', `Apfel', `Besen', `Eimer',
`Bulldozer', `Uhr', `Wolke', `Computer', `Auge' und `Blume' (siehe
autoref{Dataset images}). Die Bilder in den
drei Datensets sind gleich verarbeitet wie die Trainingsdaten (siehe
\nameref{sub:m_grund_data}). 

%todo Bild images dataset

Die Variationen (siehe \nameref{chap:m_var}) der KI umfassen zwei Umgebungen und
drei Reward-Functions \nameref{sub:t_rl_func}. Für jede Variation ist zur
Vereinfachung eine Abkürzung definiert.
\begin{itemize}
  \item \nameref{chap:m_grund} Umgebung: Grund
  \item \nameref{sub:m_var_phy}: Physik
  \item \nameref{sub:m_var_rec}: Basis
  \item \nameref{sub:m_var_speed}: Speed
  \item \nameref{sub:m_var_rec} (von MNIST Ziffern): MNIST
\end{itemize}

Die folgenden Kombinationen an Variationen der künstlichen Intelligenz werden
ausgewertet. Diese Kombinationen stellen einzelne Versionen der KI dar
\begin{itemize}
  \item Grund-Basis
  \item Grund-MNIST
  \item Grund-Speed
  \item Grund-MNIST-Speed 
  \item Physik-Basis
  \item Physik-MNIST
  \item Physik-Speed
  \item Physik-MNIST-Speed
\end{itemize}

\subsection{Testumgebung}\label{sub:m_auswert_test}

Die Leistungen der verschiedenen Variationen der künstlichen Intelligenz werden
in einer Testumgebung (siehe \nameref{sub:t_ml_func}) ausgewertet. zwischen der
Trainingsumgebung und der Testumgebung sind drei relevante Unterschiede
erkennbar. Erstens trainiert die KI in der Testumgebung nicht. Die Testumgebung
übernimmt eine trainierte Version der KI und verändert diese während dem Test
nicht. Zweitens wählt der Agent in keinem Fall mehr eine zufällige Action.
Stattdessen wählt er immer die Action mit dem höchsten Q-Value (gleichbedeutend
mit $\epsilon = 0$) (siehe \nameref{sub:t_rl_func}). Der Dritte Unterschied
liegt in den Strichbildern, die für die künstliche Intelligenz als Vorlage
dienen. Im Test zeichnet das Computerprogramm $2000$ Bilder aus einem der drei
zur Verfügung stehenden Datensets (siehe \autoref{tab:datasets}). 

Am Ende jeder Episode (das heisst jeder Zeichnung), wird der Zahlenwert für die
verschiedenen Kriterien (siehe \nameref{chap:m_eval}) nach ihrer Definition
ausgewertet und gespeichert. Die KI zeichnet auch in der Testumgebung für $64$
Steps. Wenn eine Zeichnung der Definition entsprechend früher fertig ist (siehe
\nameref{sub:m_eval_speed}), wird die Anzahl Steps zu diesem Zeitpunkt als Wert
des Kriteriums der Geschwindigkeit gespeichert und die KI zeichnet weiter. Der
Durchschnitt aller gespeicherten Werte eines Kriteriums entspricht der Leistung
der getesteten Variation in diesem Kriterium. Das Kriterium der Erkennbarkeit
(siehe \nameref{sub:m_eval_rec}) verwendet zur Auswertung je nach dem
verwendeten Datenset im Test, dasjenige vortrainierte Modell, das auf dem selben
Datenset trainiert ist. Da das Kriterium der Erkennbarkeit jeweils den Wert $0$
oder $1$ hat, ergibt der Durchschnitt aus allen Werten dieses Kriteriums eine
prozentuale Angabe (in Dezimalform) darüber, in wie vielen Fällen das richtige
Motiv erkannt wird.



