\chapter{Zusammenfassung}\label{zusammenfassung}

Diese Arbeit untersucht, ob eine KI Vorlagen von Strichbildern wie Zahlen,
Buchstaben und andere einfache Zeichnungen so nachzeichnen kann, dass ein
zeichnender Roboter prinzipiell durch die KI bedienbar wäre. Eine weiterführende
Frage untersucht die Möglichkeiten einer generativen künstliche Intelligenz, die
selbstständig, ohne eine Vorlage, Strichbilder zeichnen kann.
 
Um die Fragestellungen zu beantworten, werden die entsprechenden künstlichen
Intelligenzen unter der Verwendung von Deep Q-Learning und einem Convolutional
Neural Network (CNN) in Python mit der Keras API entwickelt. Die nachzeichnende
KI lernt in einer Umgebung, wo sie sich wie ein virtueller Stift auf einer
Zeichenfläche bewegt und dabei Bilder von handgeschriebenen Ziffern aus dem
MNIST Datenset nachzeichnet. Die KI lernt aus Bewertungen ihrer Stiftbewegungen.
Die Bewertungen basieren auf fünf quantitativen Kriterien. Beispiele für die
Kriterien sind die prozentuale Übereinstimmung zwischen der Vorlage und dem
nachgezeichneten Bild, die Geschwindigkeit und die Erkennbarkeit, die mit einer
weiteren klassifizierenden künstlichen Intelligenz bestimmt wird. Die Werte in
diesen Kriterien machen die Ergebnisse dieser Arbeit aus. Die generative KI
erlernt zu Beginn des Trainings mit entsprechenden Vorlagen das Nachzeichnen des
gewünschten Strichbildes. Erst im späteren Verlauf des Trainings wird der KI
immer seltener eine Vorlage gezeigt, bis diese vollkommen selbstständig
zeichnet. Wenn die KI ohne eine Vorlage zeichnet, erfährt diese durch die
Einschätzung eines klassifizierenden Machine Learning Modells, ob die eigene
Zeichnung erkennbar ist.

Die nachzeichnende KI erreicht eine Übereinstimmung von $90\%$ zwischen der
Vorlage und dem nachgezeichneten Bild und stellt die Zeichnungen mit
durchschnittlich $20$ Bewegungen fertig. Die KI erzielt dabei eine vergleichbare
Leistung für verschiedene Typen von Strichbildern, obwohl diese nur auf das
Nachzeichnen von Zahlen trainiert ist. Die generative KI zeichnet das gewünschte
Strichbild in $90$ bis $100$ Prozent der Fälle so nach, dass die Zeichnung für eine
passende klassifizierende KI erkennbar ist.

Die Ergebnisse sprechen dafür, dass die nachzeichnende KI erlernt, Strichbildern
nach den vordefinierten Kriterien nachzuzeichnen. Die Strichbilder müssen
allerdings von einem bestimmten Format sein und für andere Kriterien des
Zeichnens erlaubt die KI keine Aussage. Die Bewegungen der KI sind ausserdem
simuliert und somit nicht physisch. Trotzdem ist durch die beschränkte Freiheit
der Bewegungen ein zeichnender Roboter durch sie bedienbar. Die Ergebnisse der
generativen KI zeigen, dass eine künstliche Intelligenz auch selbstständig
zeichnen kann, sofern diese im Training Beispiele des gewünschten Motivs gesehen
hat und das Nachzeichnen von diesen Motiven bereits erlernt hat. Die Leistung
der generativen KI ist allerdings direkt abhängig von der Zuverlässigkeit der
klassifizierenden KI. Falsche Einschätzungen der Zeichnungen beeinflussen die
generative KI negativ, selbst wenn diese selten vorkommen.
 
Die nachzeichnende KI kann beliebige Strichbilder nachzeichnen. Diese
Strichbilder sind allerdings klein, schwarzweiss und begrenzt detailliert.
Mit einem vielseitigeren Format könnte die KI für verschiedene Anwendungen, die
einen Nachzeichenprozess benötigen, verwendet werden. Dazu gehört unter anderem
die Umwandlung von Rastergrafiken in Vektorgrafiken. Die generative KI zeichnet
aus eigenem Antrieb und entwickelt so gewissermassen eine eigene Handschrift.
Mit dem Training auf Schriftstücke einer ausgewählten Person wäre diese KI
durchaus dazu in der Lage, die Handschrift dieser Person nachzuahmen.
Schlussendlich beschreibt diese Arbeit zwei künstliche Intelligenzen mit grossem
Entwicklungspotenzial.