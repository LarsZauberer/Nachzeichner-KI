\begin{abstract}\label{abstract} ReSketch ist eine künstliche Intelligenz, die
versucht Strichbilder, wie zum Beispiel Ziffern oder Buchstaben, auf eine
physische Weise nachzuzeichnen. Um die Frage zu beantworten, in wiefern das
möglich ist, sind definierende Kriterien des Nachzeichnes festgelegt. So soll
die künstliche Intelligenz zum Beispiel nur Bewegungen ausführen können, die auch
mit einem Stift möglich wären. Die künstliche Intelligenz erlernt das
Nachzeichnen nach diesen Kriterien durch Deep Q-Learning, einem Reinforcement
Learning Modell. Das Modell basiert auf der Arbeit hinter Doodle-SDQ
\cite{zhou_learning_2018}, erfährt aber konzeptuelle Variationen wie die
Integration einer Physiksimulation. Die künstliche Intelligenz ist auf
das Nachzeichnen von Ziffern trainiert. Ein Test dieser trainierten künstlichen
Intelligenz auf Buchstaben und andere Arten von Strichbildern führt zur Antwort
auf die Frage, ob eine künstliche Intelligenz das Nachzeichnen im Allgemeinen
erlernen kann.
\end{abstract}

\newpage

\section*{Vorwort}\label{vorwort}
Diese Arbeit ist eine Untersuchung über künstliche Intelligenz. Die
Fragestellung der Untersuchung wird mithilfe einer selbst programmierten künstlichen
Intelligenz beantwortet. 

Wir haben uns für das Thema Künstliche Intelligenz entschieden, weil damit
praktische Arbeit mit intellektueller Forschung verbunden werden kann. Das Thema
ermöglicht ausgeprägte, praktische Programmierarbeiten, was uns zuspricht.
Zusätzlich ermöglicht künstliche Intelligenz einfache Forschung. Mit einfacher
Forschung ist dabei nicht der Grad der Komplexität gemeint, sondern die Vielfalt
der Möglichkeiten. Es gibt Aspekte und Anwendungen der künstlichen Intelligenz,
die für uns zugänglich sind und noch nicht zu weit erforscht sind, um neue Ideen
zu finden. Ausserdem benötigt die Forschung an künstlicher Intelligenz nur einen
Computer. Experimente und Tests können durch Programmcode ausgeführt werden. Die
Auswertung der Experimente findet auf dem selben Computer statt und die
Genauigkeit der Ergebnisse stellt ebenfalls kein Problem dar, da der Computer
die Zahlen direkt berechnet. Der Computer ist eine optimale Umgebung für eine
erste Forschungsarbeit.

Diese Arbeit ist für uns eine erste vertiefte Erfahrung mit dem grossen Gebiet
der künstlichen Intelligenz. Wir erhoffen uns durch diese Erfahrung einen
erweiterten Horizont, neues Wissen und verbesserte Programmierkenntnisse.

Vielen Dank an unseren Betreuer, Dr. Nicolas Ruh, für die hilfreichen Vorschläge,
die ausgeprägte Beratung und das Vertrauen in uns.

