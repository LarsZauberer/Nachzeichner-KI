\begin{abstract}\label{abstract} ReSketch ist eine künstliche Intelligenz, die
Strichbilder nachzeichnen kann. Strichbilder sind beispielsweise Ziffern oder
Buchstaben. Die künstliche Intelligenz kann sich beim Zeichnen so bewegen, wie
es mit einem echten Stift möglich wäre. ReSketch funktioniert mit Deep
Q-Learning, einem Reinforcement Learning Modell. Das Modell basiert dabei auf
der Arbeit hinter Doodle-SDQ \cite{zhou_learning_2018}, erfährt aber
verschiedene Erweiterungen. Die Leistung von dem Modell wird durch vordefinierte
Kriterien evaluiert, deren Werte das Resultat dieser Arbeit ausmachen. ReSketch
erreicht eine Übereinstimmung von $90\%$ zwischen der Vorlage und dem
nachgezeichneten Bild. Ausserdem kann die KI nach dem Training beliebige Arten
von Strichbildern nachzeichnen, obwohl diese lediglich auf das Zeichnen von
Zahlen trainiert ist. Eine zweite künstliche Intelligenz, die auf der
nachzeichnenden KI basiert, entfernt sich von der ursprünglichen Aufgabe. Diese
zweite KI erlernt das selbstständige Zeichnen von einem ausgewählten Motiv, ohne
eine Vorlage davon zu erhalten. Zu diesem Zweck werden die generierten
Zeichnungen der KI mit einem Klassifizierungsmodell bewertet. Mit einem
spezifischen Training dieser generativen KI können verschiedene Handschriften
emuliert werden.

\end{abstract}

    
\newpage
    
\section*{Vorwort}\label{vorwort} Diese Arbeit ist eine Untersuchung im Bereich
der künstlichen Intelligenz. Die Fragestellung der Untersuchung wird mithilfe
einer selbst programmierten künstlichen Intelligenz beantwortet.
    
Wir haben uns für das Thema künstliche Intelligenz entschieden, weil dabei
praktische Arbeit mit intellektueller Forschung verbunden wird. Das Thema
ermöglicht ausgeprägte, praktische Programmierarbeiten, was uns zuspricht.
Zusätzlich ermöglicht künstliche Intelligenz einfache Forschung. Mit einfacher
Forschung ist dabei nicht der Grad der Komplexität gemeint, sondern die Vielfalt
der Möglichkeiten. Es gibt Aspekte und Anwendungen der künstlichen Intelligenz,
die für Schüler zugänglich sind und noch nicht zu weit erforscht sind, um neue
Ideen zu finden. Ausserdem benötigt die Forschung an künstlicher Intelligenz nur
einen Computer. Experimente und Tests können durch Programmcode ausgeführt
werden. Die Auswertung der Experimente findet auf demselben Computer statt und
die Genauigkeit der Ergebnisse stellt ebenfalls kein Problem dar, da der
Computer die Zahlen direkt berechnet. Der Computer ist eine optimale Umgebung
für eine erste Forschungsarbeit.
    
Diese Arbeit ist für uns eine erste vertiefte Erfahrung mit dem Gebiet der
künstlichen Intelligenz. Wir erhoffen uns durch diese Erfahrung einen
erweiterten Horizont, neues Wissen und verbesserte Programmierkenntnisse.

Der gesamte Code der künstlichen Intelligenz mit all ihren Komponenten und Versionen ist in dem folgenden
GitHub Repository einsehbar:
\url{https://github.com/LarsZauberer/ReSketch}.

    
Vielen Dank an unseren Betreuer der Maturarbeit, Dr.\ Nicolas Ruh und an unseren
Experten im Rahmen des nationalen Wettbewerbs von Schweizer Jugend Forscht, Dr.\
Michael Tschannen, für die hilfreichen Vorschläge, die ausgeprägte Beratung und
das an uns weitergegebene technische Wissen. Vielen Dank auch an Dieter Koch für
die Zweitbeurteilung und an Günther Wasser für das Korrekturlesen dieser Arbeit.

