\chapter{Writing scientific texts in English}

This chapter was originally a separate document written by Reto
Spöhel.  It is reprinted here so that the template can serve as a
quick guide to thesis writing, and to provide some more example
material to give you a feeling for good typesetting.

% We're going to need an extra theorem-like environment for this
% chapter
\theoremstyle{plain}
\theoremsymbol{}
\newtheorem{Rule}[theorem]{Rule}

\section{Basic writing rules}

The following rules need little further explanation; they are best
understood by looking at the example in the booklet by Knuth et al.,
§2--§3.

\begin{Rule}
  Write texts, not chains of formulas.
\end{Rule}

More specifically, write full sentences that are logically
interconnected by phrases like `Therefore', `However', `On the other
hand', etc.\ where appropriate.

\begin{Rule}
  Displayed formulas should be embedded in your text and punctuated
  with it.
\end{Rule}

In other words, your writing should not be divided into `text parts'
and `formula parts'; instead the formulas should be tied together by
your prose such that there is a natural flow to your writing.

\section{Being nice to the reader}

Try to write your text in such a way that a reader enjoys reading
it. That's of course a lofty goal, but nevertheless one you should
aspire to!

\begin{Rule}
  Be nice to the reader.
\end{Rule}

Give some intuition or easy example for definitions and theorems which
might be hard to digest. Remind the reader of notations you introduced
many pages ago -- chances are he has forgotten them. Illustrate your
writing with diagrams and pictures where this helps the reader. Etc.

\begin{Rule}
  Organize your writing.
\end{Rule}

Think carefully about how you subdivide your thesis into chapters,
sections, and possibly subsections.  Give overviews at the beginning
of your thesis and of each chapter, so the reader knows what to
expect. In proofs, outline the main ideas before going into technical
details. Give the reader the opportunity to `catch up with you' by
summing up your findings periodically.

\emph{Useful phrases:} `So far we have shown that \ldots', `It remains
to show that \ldots', `Recall that we want to prove inequality (7), as
this will allow us to deduce that \ldots', `Thus we can conclude that
\ldots. Next, we would like to find out whether \ldots', etc.

\begin{Rule}
  Don't say the same thing twice without telling the reader that you
  are saying it twice.
\end{Rule}

Repetition of key ideas is important and helpful. However, if you
present the same idea, definition or observation twice (in the same or
different words) without telling the reader, he will be looking for
something new where there is nothing new.

\emph{Useful phrases:} `Recall that [we have seen in Chapter 5 that]
\ldots', `As argued before / in the proof of Lemma 3, \ldots', `As
mentioned in the introduction, \ldots', `In other words, \ldots', etc.

\begin{Rule}
  Don't make statements that you will justify later without telling
  the reader that you will justify them later.
\end{Rule}

This rule also applies when the justification is coming right in the
next sentence!  The reasoning should be clear: if you violate it, the
reader will lose valuable time trying to figure out on his own what
you were going to explain to him anyway.

\emph{Useful phrases:} `Next we argue that \ldots', `As we shall see,
\ldots', `We will see in the next section that \ldots, etc.


\section{A few important grammar rules}

\begin{Rule}
  \label{rule:no-comma-before-that}
  There is (almost) \emph{never} a comma before `that'.
\end{Rule}

It's really that simple. Examples:
\begin{quote}
  We assume that \ldots\\
  \emph{Wir nehmen an, dass \ldots}

  It follows that \ldots\\
  \emph{Daraus folgt, dass \ldots}

  `thrice' is a word that is seldom used.\\
  \emph{`thrice' ist ein Wort, das selten verwendet wird.}
\end{quote}
Exceptions to this rule are rare and usually pretty obvious. For
example, you may end up with a comma before `that' because `i.e.' is
spelled out as `that is':
\begin{quote}
  For \(p(n)=\log n/n\) we have \ldots{} However, if we choose \(p\) a
  little bit higher, that is \(p(n)=(1+\varepsilon)\log n/n\) for some
  \(\varepsilon>0\), we obtain that\ldots
\end{quote}
Or you may get a comma before `that' because there is some additional
information inserted in the middle of your sentence:
\begin{quote}
  Thus we found a number, namely \(n_0\), that satisfies equation (13).
\end{quote}
If the additional information is left out, the sentence has no comma:
\begin{quote}
  Thus we found a number that satisfies equation (13).
\end{quote}
(For `that' as a relative pronoun, see also
Rules~\ref{rule:non-defining-has-comma}
and~\ref{rule:defining-without-comma} below.)

\begin{Rule}
  There is usually no comma before `if'.
\end{Rule}

Example:
\begin{quote}
  A graph is not \(3\)-colorable if it contains a \(4\)-clique.\\
  \emph{Ein Graph ist nicht \(3\)-färbbar, wenn er eine \(4\)-Clique
    enthält.}
\end{quote}
However, if the `if' clause comes first, it is usually separated from
the main clause by a comma:
\begin{quote}
  If a graph contains a \(4\)-clique, it is not \(3\)-colorable .\\
  \emph{Wenn ein Graph eine \(4\)-Clique enthält, ist er nicht
    \(3\)-färbbar.}
\end{quote}

There are more exceptions to these rules than to
Rule~\ref{rule:no-comma-before-that}, which is why we are not
discussing them here. Just keep in mind: don't put a comma before `if'
without good reason.

\begin{Rule}
  \label{rule:non-defining-has-comma}
  Non-defining relative clauses have commas.
\end{Rule}
\begin{Rule}
  \label{rule:defining-without-comma}
  Defining relative clauses have no commas.
\end{Rule}

In English, it is very important to distinguish between two types of
relative clauses: defining and non-defining ones. This is a
distinction you absolutely need to understand to write scientific
texts, because mistakes in this area actually distort the meaning of
your text!

It's probably easier to explain first what a \emph{non-defining}
relative clause is. A non-defining relative clauses simply gives
additional information \emph{that could also be left out} (or given in
a separate sentence). For example, the sentence
\begin{quote}
  The \textsc{WeirdSort} algorithm, which was found by the famous
  mathematician John Doe, is theoretically best possible but difficult
  to implement in practice.
\end{quote}
would be fully understandable if the relative clause were left out
completely. It could also be rephrased as two separate sentences:
\begin{quote}
  The \textsc{WeirdSort} algorithm is theoretically best possible but
  difficult to implement in practice. [By the way,] \textsc{WeirdSort}
  was found by the famous mathematician John Doe.
\end{quote}
This is what a non-defining relative clause is. \emph{Non-defining
  relative clauses are always written with commas.} As a corollary we
obtain that you cannot use `that' in non-defining relative clauses
(see Rule~\ref{rule:no-comma-before-that}!). It would be wrong to
write
\begin{quote}
  \st{The \textsc{WeirdSort} algorithm, that was found by the famous
    mathematician John Doe, is theoretically best possible but
    difficult to implement in practice.}
\end{quote}
A special case that warrants its own example is when `which' is
referring to the entire preceding sentence:
\begin{quote}
  Thus inequality (7) is true, which implies that the Riemann
  hypothesis holds.
\end{quote}
As before, this is a non-defining relative sentence (it could be left
out) and therefore needs a comma.

So let's discuss \emph{defining} relative clauses next. A defining
relative clause tells the reader \emph{which specific item the main
  clause is talking about}. Leaving it out either changes the meaning
of the sentence or renders it incomprehensible altogether.  Consider
the following example:

\begin{quote}
  The \textsc{WeirdSort} algorithm is difficult to implement in
  practice. In contrast, the algorithm that we suggest is very simple.
\end{quote}

Here the relative clause `that we suggest' cannot be left out -- the
remaining sentence would make no sense since the reader would not know
which algorithm it is talking about. This is what a defining relative
clause is. \textit{Defining relative clauses are never written with
  commas.} Usually, you can use both `that' and `which' in defining
relative clauses, although in many cases `that' sounds better.

As a final example, consider the following sentence:
\begin{quote}
  For the elements in \(\mathcal{B}\) which satisfy property (A), we
  know that equation (37) holds.
\end{quote}
This sentence does not make a statement about all elements in
\(\mathcal{B}\), only about those satisfying property (A). The relative
clause is \emph{defining}. (Thus we could also use `that' in place of
`which'.)

In contrast, if we add a comma the sentence reads
\begin{quote}
  For the elements in \(\mathcal{B}\), which satisfy property (A), we
  know that equation (37) holds.
\end{quote}

Now the relative clause is \emph{non-defining} -- it just mentions in
passing that all elements in \(\mathcal{B}\) satisfy property (A). The
main clause states that equation (37) holds for \emph{all} elements in
\(\mathcal{B}\). See the difference?


\section[Things you (usually) don't say in English]%
{Things you (usually) don't say in English -- and what to say
  instead}
\label{sec:list}

Table~\ref{tab:things-you-dont-say} lists some common mistakes and
alternatives.  The entries should not be taken as gospel -- they don't
necessarily mean that a given word or formulation is wrong under all
circumstances (obviously, this depends a lot on the context). However,
in nine out of ten instances the suggested alternative is the better
word to use.

\begin{table}
  \centering
  \caption{Things you (usually) don't say}
  \label{tab:things-you-dont-say}
  \begin{tabular}{lll}
    \toprule
    \st{It holds (that) \dots} & We have \dots & \emph{Es gilt \dots}\\
    \multicolumn{3}{l}{\quad\footnotesize(`Equation (5) holds.' is fine, though.)}\\
    \st{$x$ fulfills property $\mathcal{P}$.}& \(x\) satisfies property \(\mathcal{P}\). & \emph{\(x\) erfüllt Eigenschaft \(\mathcal{P}\).} \\
    \st{in average} & on average & \emph{im Durchschnitt}\\
    \st{estimation} & estimate   & \emph{Abschätzung}\\
    \st{composed number} & composite number & \emph{zusammengesetzte Zahl}\\
    \st{with the help of} & using & \emph{mit Hilfe von}\\
    \st{surely} & clearly & \emph{sicher, bestimmt}\\
    \st{monotonously increasing} & monotonically incr. & \emph{monoton steigend}\\
    \multicolumn{3}{l}{\quad\footnotesize(Actually, in most cases `increasing' is just fine.)}\\
    \bottomrule
  \end{tabular}
\end{table}

%%% Local Variables:
%%% mode: latex
%%% TeX-master: "thesis"
%%% End:
